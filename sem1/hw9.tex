\documentclass[letterpaper,12pt]{article}

\usepackage{amsmath,amsfonts,amsthm,amssymb}
\usepackage{mathrsfs} %script font
\usepackage{enumerate}

\usepackage[margin=1in]{geometry}

%\usepackage{parskip}
%\parskip=1.5\baselineskip

\newcommand{\io}{\;\text{i.o.}}
\renewcommand{\aa}{\;\text{a.a.}}
\newcommand{\Ft}{F^\sim}
\newcommand{\N}{\mathbf{N}}
\newcommand{\E}{\mathscr{E}}
\newcommand{\p}{\mathscr{P}}
\newcommand{\ap}{{\alpha'}}

%\def \B {\mathscr{B}}
%\def \F {\mathscr{F}}
%\def \Q {\mathbb{Q}}
%\def \E {\mathbb{E}}
%\def \R {\mathbf{R}}

\begin{document}

\textsf{
\begin{flushleft}
\sc James K. Pringle \\
\normalfont 550.620 \\
Dr. Jim Fill \\
Assignment 9 \\
7 December 2012, Friday
\end{flushleft}
} 
\bigskip

\begin{center}
\bf Homework \#9 \\
Chung 4.4.6
\end{center}

Let the r.v.'s $\{X_\alpha \}$ have the p.m.'s $\{\mu_\alpha \}$. 
If for some real $r > 0$, $\E\{|X_\alpha|^r\}$ is bounded in $\alpha$, then $\{\mu_\alpha \}$ is tight.

\begin{proof}
Let the r.v.'s $\{X_\alpha \}$ have the p.m.'s $\{\mu_\alpha \}$.
Let $\E\{|X_\alpha|^r\}$ be bounded in $\alpha$ for some real $r > 0$. 
Hence, for all $\alpha$, from their index set $A$, we have
\begin{equation} \label{CONTRADICTION}
0 \leq \E\{|X_\alpha|^r\} = \int |X|^r d \mu_\alpha < M
\end{equation}
for some finite, positive $M$.

Suppose by way of contradiction that $\{\mu_\alpha \}$ is not tight.
It follows from negating the definition of tightness (Chung 94) that there exists some $\epsilon > 0$ such that for all finite intervals $I$ we have
\begin{equation} 
\label{star}
\inf_{\alpha \in A} \mu_\alpha(I) \leq 1 - \epsilon 
\text{.}
\end{equation}
Since $\{\mu_\alpha \}$ are probability measures, $\mu_\alpha(I) = 1 - \mu_\alpha(I^c)$. Remember that for real-valued sets $S$, it is true that $\inf -S = - \sup S$. 
Hence $\eqref{star}$ becomes
\begin{align*}
\inf_{\alpha \in A} \mu_\alpha(I) &\leq 1 - \epsilon \\
\inf_{\alpha \in A} (1 - \mu_\alpha(I ^ c)) &\leq 1 - \epsilon \\
1 + \inf_{\alpha \in A} - \mu_\alpha(I ^ c) &\leq 1 - \epsilon \\
1 - \sup_{\alpha \in A} \mu_\alpha(I^c)  &\leq 1 - \epsilon \\
\sup_{\alpha \in A} \mu_\alpha(I^c)  &\geq \epsilon
\text{.}
\end{align*}
Therefore, there must be some $\ap \in A$, such that 
\begin{equation} 
\label{cross}
\mu_{\ap}(I^c) > \epsilon/2
\end{equation}
for all finite intervals $I$.

Let $I$ be the finite interval $I_b = (-b,b)$ for a positive real $b$. 
Thus $I_b^c = (-\infty, -b] \cup [b, \infty)$. 
Notice that $\p\{|X_\ap| \geq b \} = \mu_\ap(I_b^c)$.
Using inequality $\eqref{cross}$ and applying Chebyshev's inequality (Chung 51),
\[
\frac{\epsilon}{2} <\mu_\ap(I_b^c) = \p\{|X_\ap| \geq b \} \leq \frac{\E \{|X_\ap|^r \}}{b^r}
\text{.}
\]
After rearranging,
\begin{equation}
\label{fin}
\frac{\epsilon b^r}{2} < \E \{|X_\ap|^r \} 
\text{.}
\end{equation}
Since $b$ is restricted to be positive real choose $b = (2M / \epsilon)^{1/r}$. 
Now $\eqref{fin}$ becomes
\[
M  
= \frac{\epsilon}{2}\frac{2M}{\epsilon} 
= \frac{\epsilon}{2} \left(\left(\frac{2M}{\epsilon}\right)^{1/r}\right)^r
= \frac{\epsilon b^r}{2} 
< \E \{|X_\ap|^r \} 
\text{,}
\]
a contradiction of inequality $\eqref{CONTRADICTION}$.

Therefore, given that for some real $r > 0$, $\E\{|X_\alpha|^r\}$ is bounded in $\alpha$, we conclude that $\{\mu_\alpha \}$ is tight. 
\end{proof}

\section*{Acknowledgments}
I thank Leonardo Collado Torres for being able to work with him on this problem.

\end{document}