\documentclass[letterpaper,10pt]{article}
%\usepackage[utf8x]{inputenc}

\usepackage{amsmath,amsfonts,amsthm,amssymb}
\usepackage{mathrsfs} %script font

%\parindent=0pt
%\parskip=10pt
\usepackage[margin=1in]{geometry}

\usepackage{parskip}
\parskip=1.5\baselineskip

\newcommand{\io}{\;\text{i.o.}}
\renewcommand{\aa}{\;\text{a.a.}}
\newcommand{\Ft}{F^\sim}
\newcommand{\N}{\mathbf{N}}

\def \B {\mathscr{B}}
\def \F {\mathscr{F}}
\def \Q {\mathbb{Q}}
\def \R {\mathbf{R}}

\begin{document}

\thispagestyle{empty}

\textsf{
\begin{flushleft}
\sc James K. Pringle \\
\normalfont 550.620 \\
Dr. Jim Fill \\
Assignment 6 \\
14 November 2012, Wednesday
\end{flushleft}
} \bigskip

\begin{center}
\bf Homework \#6 (to turn in)
\end{center}

\begin{enumerate}
\item
Use Fatou's Lemma to prove Pratt's Theorem.
\begin{proof}
First we prove a lemma to this proof. 
\textbf{Lemma:} Let $x_n \rightarrow x$ and $\{y_n\}$ be some series not necessarily convergent. Then 
\[
\liminf_n (x_n + y_n) = x + \liminf_n y_n \text{.}
\]
It is clear that $\inf_{ n > m} (x_n + y_n) \geq \inf_{n >m} (x_n) + \inf_{n > m} (y_n)$. Taking the limit as $m$ approaches infinity, the definition of $\liminf$, we have 
\[
\liminf_n (x_n + y_n) \geq \liminf (x_n) + \liminf_n y_n = x + \liminf_n y_n \text{.}
\]
Now, $\liminf$ always exists as some number in the extended real line. Let $\liminf_n y_n = l$. 
From the properties of $\liminf$ we know that there exists some subsequence $\{y_{n_k}\} \subset \{y_n \}$ such that $\{y_{n_k} \}$ converges to $l$. 
Furthermore, if we ``thin out'' the sequence $\{x_n + y_n\}$ we can only increase the $\liminf$ of that sequence.
Hence
\[
\liminf_n (x_n + y_n) \leq \liminf_n (x_{n_k} + y_{n_k}) = \lim_n (x_{n_k} + y_{n_k}) = x + \liminf_n (y_n) \text{.}
\]
Thus, we have $\liminf_n (x_n + y_n) = x + \liminf_n y_n$ as desired. This completes the lemma.

Now we move on to the proof of the homework problem. We assume Fatou's Lemma to be true. We also assume the hypotheses of Pratt's Theorem, which are threefold. First, $L_n \rightarrow L$, $X_n \rightarrow X$ and $U_n \rightarrow U$. Notice those are all random variables. Second, for all $n$ we have $L_n \leq X_n \leq U_n$. Third, $L_n, L, U_n, U$ are integrable and $\int L_n \rightarrow \int L$ and $\int U_n \rightarrow \int U$. 

We show two things. The first is that $X_n$ and $X$ are integrable. And here is the proof. 
By hypothesis, $X_n \leq U_n$. Thus $X_n^+ \leq U_n^+$. By the properties of integral, it follows $\int X_n^+ \leq \int U_n^+ < \infty$. 
By hypothesis, $X_n^- \leq L_n^-$ and consequently $\int X_n^- \leq \int L_n^- < \infty$. 
Therefore, since the integral of its positive and negative part is finite, $X_n$ is integrable. 
Taking the limit of the inequality $L_n \leq X_n \leq U_n$ we see $L \leq X \leq U$. 
Using the exact same argument above without subscripts allows us to conclude that $X$ is integrable.

The second thing we show is that $\int X_n \rightarrow \int X$. Since $U_n - X_n \geq 0$, it follows $U_n - X_n \in Q_-$. Thus we can use our lemma from above to get
\[
\int U - \int \limsup X_n = \int U + \int \liminf (- X_n) = \int ( U + \liminf (- X_n) )
= \int \liminf (U_n - X_n)\text{.}
\]
Starting with what we have from above and using Fatou's Lemma we have
\begin{align*}
\int U - \int \limsup X_n &= \int \liminf (U_n - X_n) \\
&\leq \liminf \int (U_n - X_n) \\
&= \liminf (\int U_n - \int X_n) \\
&= \int U - \limsup (\int X_n)
\end{align*}
that last equality coming by our lemma. Thus 
\[
\limsup (\int X_n) \leq \int \limsup X_n \text{.}
\]
Next we do nearly identical calculations on $X_n - L_n$ since it is also nonnegative and hence in $Q_-$. Also note that since $L_n \rightarrow L$ and $\int L_n \rightarrow \int L$, clearly $-L_n \rightarrow -L $ and $\int -L_n \rightarrow \int -L$.
Using our lemma from above we get
\[
\int \liminf (X_n) + \int (-L) = \int (\liminf(X_n) + (- L))
= \int \liminf (X_n - L_n)\text{.}
\]
Continuing using Fatou's Lemma we have 
\begin{align*}
\int \liminf (X_n) + \int (-L) &= \int \liminf (X_n - L_n) \\
& \leq \liminf \int (X_n - L_n) \\
&= \liminf ( \int (X_n) + \int (-L_n) ) \\
&= \liminf \int (X_n) + \int (-L)
\end{align*}
that last equality coming by our lemma. Thus
\[
\int \liminf (X_n) \leq \liminf (\int X_n) \text{.}
\]
Since the $\liminf$ is always less than or equal to the $\limsup$ of any given series we have
\[
\int \liminf (X_n) \leq \liminf (\int X_n) \leq \limsup (\int X_n) \leq \int \limsup X_n \text{.}
\]
By hypothesis $X_n \rightarrow X$. Thus we can rewrite the above statement to be
\[
\int X \leq \liminf (\int X_n) \leq \limsup (\int X_n) \leq \int X \text{.}
\]
And hence $\liminf (\int X_n) = \limsup (\int X_n) = \int X$. Since the $\liminf$ and the $\limsup$ are equal, they are both equal to the limit of $\int X_n$. Finally, we can conclude that $\lim \int X_n = \int X$, which finishes the proof of Pratt's Theorem. 
\end{proof}


\item
Use Pratt's Theorem to prove Dominated Convergence Theorem.
\begin{proof}
We assume Pratt's Theorem to be true. We also assume the hypotheses of Dominated Convergence Theorem. 
Namely, let $\sup_n | X_n |$ be integrable and $X_n \rightarrow X$. From the definition of $\sup$ we have $\sup_n | X_n | \leq X_n \leq \sup_n | X_n |$ and $\sup_n | X_n | \leq X \leq \sup_n | X_n |$. 
Clearly $X$ is bounded above by $\sup_n | X_n |$ and below by $-\sup_n | X_n |$. 
Both bounds are integrable, therefore $X$ is integrable. In order to apply Pratt's Theorem, we take our lower bound series of random variables to be the constant series with the element $-\sup_n | X_n |$ and the upper bound series to be the constant series with the element $\sup_n | X_n |$. 
It is obvious that all the hypotheses of Pratt's Theorem hold, so its conclusion $\int X_n \rightarrow \int X$ is true. 
This finishes the proof of Dominated Convergence Theorem.
\end{proof}
\end{enumerate}

\end{document}
