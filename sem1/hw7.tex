\documentclass[letterpaper,12pt]{article}

\usepackage{amsmath,amsfonts,amsthm,amssymb}
\usepackage{mathrsfs} %script font
\usepackage{enumerate}

\usepackage[margin=1in]{geometry}

\usepackage{parskip}
%\parskip=1.5\baselineskip

\newcommand{\io}{\;\text{i.o.}}
\renewcommand{\aa}{\;\text{a.a.}}
\newcommand{\Ft}{F^\sim}
\newcommand{\N}{\mathbf{N}}

\def \B {\mathscr{B}}
\def \F {\mathscr{F}}
\def \Q {\mathbb{Q}}
\def \E {\mathbb{E}}
\def \R {\mathbf{R}}

\begin{document}

\textsf{
\begin{flushleft}
\sc James K. Pringle \\
\normalfont 550.620 \\
Dr. Jim Fill \\
Assignment 7 \\
26 November 2012, Monday
\end{flushleft}
} \bigskip

\begin{center}
\bf Homework \#7
\end{center}

\begin{enumerate}[(a)]
\item
Prove that $\|X\|_p$ increases with $0 < p \leq \infty$.

\begin{proof}
Jensen's inequality states that if $\varphi$ is a convex function and $X$ is a random variable, we have 
\[
\varphi (\E [ X ]) \leq \E [ \varphi(X) ].
\]
Let $0 < p < q <\infty$. Let $\varphi (x) = |x|^{q/p}$. 
It follows that $\varphi$ is a convex function. 
Calculating, we see 
\[
\|X\|_p  = (\E |X|^p)^{1/p} = (\varphi (\E |X|^p))^{1/q} \leq (\E \varphi(|X|^p))^{1/q} = (\E |X|^q)^{1/q} = \|X\|_q
\text{.}
\]
This shows that the $L^p$ norm is increasing on $0 < p < \infty$. 
Now we consider $p=\infty$. Let $S = \{\omega : X(\omega) > \|X\|_\infty\}$. 
From the definition of essential supremum, it follows that $P(S) = 0$ or that $S$ is a null set. For any $p$ with $0 < p < \infty$, we have 
\[
\|X\|_p  = (\int |X|^p)^{1/p} = (\int_{\Omega \setminus S} |X|^p)^{1/p} \leq (\int_{\Omega \setminus S} \|X\|_\infty^p)^{1/p} = (\|X\|_\infty^p P(\Omega \setminus S))^{1/p} = \|X\|_\infty.
\]
Hence $L^p$ norm is increasing on $0 < p \leq \infty$.
\end{proof}

\item
Under what conditions does it happen that $0 < r < s \leq \infty$ and $\|X\|_r = \|X\|_s < +\infty$?

\item
Prove that the spaces $L^p$ decrease with $0 < p \leq \infty$. Under what conditions do $L^r$ and $L^s$ with $r < s$ contain exactly the same r.v.'s?

\begin{proof}
Let $0 < q < r \leq \infty$. Let $X \in L^r$. It follows that $\| X \|_r < \infty$.
By the increasingness of the  $L^p$ norm, $\| X \|_q < \infty$. Hence $X \in L^q$.
Therefore, $L^r \subset L^q$ and the $L^p$ spaces are decreasing.
\end{proof}

\item
Let $S = \{ p: 0 < p <\infty \text{ and } \|X\|_p < +\infty\}$. 
Show that $S$ is of the form $S = (0,p_0)$ or $S = (0,p_0]$ for some $0 \leq p_0 \leq \infty$.

\begin{proof}
This follows from the increasingness of $L^p$. It cannot be any different. Examine $S^c$. By the increasingness of $L^p$, we have $S^c$ must be of the form $(p_0, \infty]$ or $[p_0, \infty]$. Hence $S$ must be of the form $(0, p_0)$ or $(0,p_0]$.
\end{proof}

\item
Prove that $\log(\|X\|_p^p)$ is convex in $p \in \text{interior}(S)$ and that $\|X\|_p$ is continuous in $p \in S$.

\item
Show that $\|X\|_\infty = \lim_{p \uparrow \infty} \uparrow \|X\|_p$.

\begin{proof}
We already know that the $L^p$ norm increases. Let $\epsilon > 0$ and 
$S_\epsilon = \{ \omega : \|X\|_\infty - \epsilon \leq X(\omega) \leq \|X\|_\infty \}$. By the normality and right-continuity of $F_X$, we have that $P(S_\epsilon) = \alpha > 0$. It follows that 
\[
\int |X|^p \geq \int_{S_\epsilon} |X|^p \geq \int_{S_\epsilon} (\|X\|_\infty - \epsilon)^p = \alpha (\|X\|_\infty - \epsilon)^p.
\]
For sufficiently small $\epsilon$ and sufficiently large $p$, we have 
\[
\int |X|^p \geq (\|X\|_\infty - \epsilon)^p.
\]
Taking the $1/p$-th power and the limit as $p$ tends to infinity we have 
\[
\lim_{p \uparrow \infty} \uparrow \|X\|_p \geq \|X\|_\infty - \epsilon.
\]
Since $\epsilon$ is arbitrary, we have shown $\|X\|_\infty = \lim_{p \uparrow \infty} \uparrow \|X\|_p$.
\end{proof}

\item
Assume that $S \neq \emptyset$ and prove that 
\[
\lim_{p \downarrow 0} \downarrow \|X\|_p = \exp\{E\log|X|\},
\]
with the understanding that $\exp\{-\infty\} := 0$.

\end{enumerate}

























\end{document}