\documentclass[letterpaper,10pt]{article}

\usepackage{amsmath,amsfonts,amsthm,amssymb}
\usepackage[margin=1in]{geometry}
\usepackage{mathrsfs} %script font

\def \B {\mathscr{B}}
\def \F {\mathscr{F}}
\def \Q {\mathbb{Q}}
\def \R {\mathbf{R}}

\begin{document}

\textsf{
\begin{flushleft}
\sc James K. Pringle \\
\normalfont 550.620 \\
Dr. Jim Fill \\
Assignment 5 \\
24 October 2012, Wednesday
\end{flushleft}
} \bigskip

\begin{center}
\bf Problem 13.3
\end{center}

\begin{enumerate}
\item[(13.3)]
Suppose that $f : \Omega \rightarrow R^1$. Show that $f$ is measurable $T^{-1}\F'$ if and only if there exists a map $\varphi: \Omega' \rightarrow R'$ such that $\varphi$ is measurable $\F'$ and $f = \varphi T$. 
\textit{Hint:} First consider simple functions and then use Theorem 13.5.

\begin{proof}
First we prove the backward implication. Let $\varphi: \Omega' \rightarrow R'$ such that $\varphi$ is measurable $\F'$ and $f = \varphi T$. 
By construction $T$ is measurable $T^{-1}\F' / \F'$ and by assumption, $\varphi$ is measurable $\F' / R^1$. 
Hence, by theorem 13.1 (ii), $f = \varphi T$ is measurable $T^{-1} \F' / R^1$ as desired.

Now we show the forward implication. Let $f$ be measurable $T^{-1}\F'$. The rest of this proof comes from "Notes on the Problems" in Billingsley. 
By Theorem 13.5, there exist simple functions $f_n$, measurable $T^{-1}\F'$, such that $f_n(\omega) \rightarrow f(\omega)$ for each $\omega$. 
Since $f_n$ is simple and measurable $T^{-1}\F'$, we can write $f_n = \sum_i x_{ni} I_{A_{ni}}$ with $A_{ni} \in T^{-1}\F'$. 
Take $A_{ni}' \in \F'$ so that $A_i = T^{-1}A_{ni}'$ and set $\varphi_n = \sum_i x_{ni} I_{A_{ni}'}$.
By construction $\varphi_n$ is measurable $\F'$.
It follows that $f_n = \varphi_n T$ for all $n$.
Taking the limit of both sides, we have $f = \varphi T$.
Let $C'$ be the set of $\omega'$ for which $\varphi_n(\omega')$ has a finite limit, and define $\varphi(\omega') = \lim_n \varphi_n(\omega')$ for $\omega' \in C'$ and $\varphi(\omega') = 0$ for $\omega' \notin C'$.
Since $f_n \rightarrow f$ for all $\omega \in \Omega$, it follows that $T(\Omega) \subset C'$. Finally, $C' \in \F'$, therefore, $\varphi$ is measurable $\F'$ by Theorem 13.4 (iii).
\end{proof}

\end{enumerate}

\end{document}