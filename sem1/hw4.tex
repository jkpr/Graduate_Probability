\documentclass[letterpaper,10pt]{article}
\usepackage[utf8x]{inputenc}

\usepackage{amsmath,amsfonts,amsthm,amssymb}
\usepackage{mathrsfs} %script font

\parindent=0pt
\parskip=10pt
\usepackage[margin=1in]{geometry}

\newcommand{\io}{\;\text{i.o.}}
\renewcommand{\aa}{\;\text{a.a.}}
\newcommand{\Ft}{F^\sim}
\newcommand{\N}{\mathbf{N}}

\def \B {\mathscr{B}}
\def \F {\mathscr{F}}
\def \Q {\mathbb{Q}}
\def \R {\mathbf{R}}

\begin{document}

\thispagestyle{empty}

\textsf{
\begin{flushleft}
\sc James K. Pringle \\
\normalfont 550.620 \\
Dr. Jim Fill \\
Assignment 4 \\
17 October 2012, Wednesday
\end{flushleft}
} \bigskip

\begin{center}
\bf Homework \#4 (to turn in)
\end{center}


Define an ``inverse'' $\Ft$ to $F$ by the recipe
\[
\Ft(t) := \inf\{x : F(x) \geq t\} \text{, } 0  < t < 1
\]
Show that $\Ft$ is 
\begin{enumerate}
\item[(a)]
increasing
\item[(b)]
\textit{left} continuous
\item[(c)]
a map of $(0,1)$ into $\R$.
\end{enumerate}
\textit{Solution:}
\begin{enumerate}
\item[(a)]
We show $\Ft$ is increasing (or more precisely, non-decreasing). Let $s,t \in (0,1)$ and assume $s \leq t$. 
Define $A = \{x : F(x) \geq s\}$ and $B = \{ x : F(x) \geq t\}$. Clearly, if $x \in B$, then $F(x) \geq t \geq s$. It follows that $x \in A$. 
Thus $B \subset A$. Thus $\inf A$ is a lower bound of $B$. Since $\inf B$ is the largest lower bound, $\inf B \geq \inf A$. This is equivalent to $\Ft(t) \geq \Ft(s)$. 
Therefore $t \geq s$ implies $\Ft(t) \geq \Ft(s)$, showing $\Ft$ is increasing.
\mbox{}~\hfill $\Box$
\item[(b)]
Here we show $\Ft$ is left continuous. 
Let $\{t_n\} \subset (0,1)$ with $t_n \uparrow t$, by which we mean that $\{t_n\}$ is a monotone increasing sequence with limit $t$. 
Since $\Ft$ is increasing by (a), it follows that $\{ \Ft(t_n)\}$ is an increasing sequence with $\Ft(t_n) \leq \Ft(t)$ for all integer $n$. Note $\Ft(t)$ is finite by (c).
Furthermore, since $\{ \Ft(t_n)\}$ is a bounded and monotone sequence, it has a limit, call it $m$, according to the Monotone Convergence Theorem. 
Taking the limit of $\Ft(t_n) \leq \Ft(t)$, we have
\begin{align*}
\lim_n F(t_n) &\leq \lim_n \Ft(t) \\
m &\leq \Ft(t) \text{.}
\end{align*}
We now show $\Ft(t) \leq m$ to conclude $\Ft(t) = m = \lim_n \Ft(t_n)$. Equation (2) of the handout states 
\[
F(\Ft(t)) \geq t \text{.}
\]
Hence for all $n$, $F(\Ft(t_n)) \geq t_n$. 
Because $\Ft$ is increasing, $\Ft(t_n) \uparrow m$, and it follows that for all $n$, $\Ft(t_n) \leq m$. 
Furthermore, since $F$ is increasing, $F(\Ft(t_n)) \leq F(m)$. Linking the two inequalities together, it is clear $t_n \leq F(\Ft(t_n)) \leq F(m)$.
Taking the limit we have
\begin{align*}
\lim_n t_n &\leq \lim_n F(m) \\
t &\leq F(m) \text{.}
\end{align*}
Hence $m \in \{x : F(x) \geq t\}$, and $\inf \{x : F(x) \geq t\} = \Ft(t) \leq m$. Therefore, we conclude $\lim_n \Ft(t_n) = \Ft(t)$, and we have demonstrated left continuity for $\Ft$.
\mbox{}~\hfill $\Box$
\item[(c)]
Finally we show $\Ft$ is a map of $(0,1)$ into $\R$.
It is obvious that $\Ft$ is either a real number or infinite.
Let $t \in (0,1)$. 
Since $F$ is normalized, it is clear that there exists $x_1$ such that $0< F(x_1) < t$. 
Since $F$ is increasing, $x_1$ is a lower bound to $\{x : F(x) \geq t\}$. Hence $x_1 \leq \inf \{x : F(x) \geq t\} = \Ft(t)$. 
On the other hand, there exists $x_2$ such that $1 > F(x_2) > t$ because $F$ is normalized. 
Clearly, $x_2 \in \{x : F(x) \geq t\}$, and $x_2 \geq \inf \{x : F(x) \geq t\} = \Ft(t)$. 
Combining these inequalities, we have $x_1 \leq \Ft(t) \leq x_2$, which shows that $\Ft(t)$ is finite. 
Hence $\Ft$ is a map of $(0,1)$ into $\R$.
\mbox{}~\hfill $\Box$
\end{enumerate}
From the text, "one has the important switching relation $t \leq F(x) \Leftrightarrow \Ft(t) \leq x$ \dots Supply the details." In other words, we prove that $t \leq F(x)$ if and only if $\Ft(t) \leq x$.

\textit{Solution: }
First we prove the forward implication. Assume $t \leq F(x)$. Automatically $x \in \{y : F(y) \geq t\}$. Thus $x \geq \inf \{y : F(y) \geq t\} = \Ft(t)$, completing the proof of the forward implication. 
Now we prove the backward implication.  
Assume $\Ft(t) \leq x$. Since $F$ is increasing, we have $F(\Ft(t)) \leq F(x)$. 
By (2) in the handout, we have $F(\Ft(t)) \geq t$.
Thus $t \leq F(\Ft(t)) \leq F(x)$. This completes the proof of the biconditional. 
\mbox{}~\hfill $\Box$

\end{document}
