\documentclass[letterpaper,12pt]{article}

\usepackage{amsmath,amsfonts,amsthm,amssymb}
\usepackage{mathrsfs} %script font
\usepackage{enumerate}

\usepackage[margin=1in]{geometry}

\usepackage{parskip}
%\parskip=1.5\baselineskip

\newcommand{\io}{\;\text{i.o.}}
\renewcommand{\aa}{\;\text{a.a.}}
\newcommand{\Ft}{F^\sim}
\newcommand{\N}{\mathbf{N}}

\def \B {\mathscr{B}}
\def \F {\mathscr{F}}
\def \Q {\mathbb{Q}}
\def \E {\mathbb{E}}
\def \R {\mathbf{R}}

\begin{document}

\textsf{
\begin{flushleft}
\sc James K. Pringle \\
\normalfont 550.620 \\
Dr. Jim Fill \\
Assignment 8 \\
3 December 2012, Monday
\end{flushleft}
} 
\bigskip

\begin{center}
\bf Homework \#8
\end{center}

If a sequence of p.m.'s converges vaguely to an atomless p.m., then the convergence is uniform for all intervals, finite or infinite.

\begin{proof}
Let $\{\mu_n\}$ be a sequence of probability measures that converge vaguely to an atomless probability measure $\mu$. Since $\mu$ is atomless, we have that for all intervals $(a,b]$ with finite $a$ and $b$,
\[
\mu_n(a,b] \rightarrow \mu(a,b]
\]
\end{proof}

\end{document}