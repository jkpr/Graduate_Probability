% /**
%  * A template for solutions to textbook problems. 
%  *
%  * The packages and newcommands are a good starting point.
%  *
%  * Author: James K. Pringle
%  * E-mail: jameskpringle@gmail.com
%  * Last Changed: 5 September 2013
%  *
%  * "LaTeX countains the increasing union of MS Word"
%  */
%~~~~~~~~~~~~~~~~~~~~~~~~~~~~~~~~~~~~~~~~~~~~~~~~~~~~~~~~~%
%%%%%%%%%%%%%%%%%%%%%%%%%%%%%%%%%%%%%%%%%%%%%%%%%%%%%%%%%%%
%                                                         %
%                        PAGE SETUP                       %
%                                                         %
%%%%%%%%%%%%%%%%%%%%%%%%%%%%%%%%%%%%%%%%%%%%%%%%%%%%%%%%%%%
\documentclass[letterpaper, 12pt]{article}

% 1in margins all the way around
\usepackage[margin=1in]{geometry}

% Sets \parindent to 0 and \parskip to stretchable.
\usepackage{parskip}
% Use for bigger spaces between paragraphs.
%\parskip=1.5\baselineskip

% Set headers and footers
\usepackage{fancyhdr}
\pagestyle{fancy}
% Header
\renewcommand{\headrulewidth}{0.4pt}
\lhead{\textsc{\textauthor}}
\chead{\textsc{\thedate}}
\rhead{\textsc{\mynamehdr}}
% Footer
\renewcommand{\footrulewidth}{0.4pt}
\lfoot{}
\cfoot{\thepage}
\rfoot{}

% Make the space between lines slightly more generous 
% than normal single spacing, but compensate so that the 
% spacing between rows of matrices still looks normal.  
% Note that 1.1=1/.9090909...
\renewcommand{\baselinestretch}{1.1}
\renewcommand{\arraystretch}{.91}

%%%%%%%%%%%%%%%%%%%%%%%%%%%%%%%%%%%%%%%%%%%%%%%%%%%%%%%%%%%
%                                                         %
%                      USEFUL PACKAGES                    %
%                                                         %
%%%%%%%%%%%%%%%%%%%%%%%%%%%%%%%%%%%%%%%%%%%%%%%%%%%%%%%%%%%

% The classic three
\usepackage{amsmath,amsthm,amssymb}

% Define \newtheorem for use
% No numbers, labeled 'Theorem'
\newtheorem*{nthm}{Theorem}

% Not sure what this is for
\usepackage{amsfonts}

% Fancy script font
\usepackage{mathrsfs}

% Makes enumerate environment much easier to customize
% by specifying the counter
\usepackage{enumerate}

% Color
\usepackage{color}
\usepackage[usenames,dvipsnames,svgnames,table]{xcolor}

% URL links
\usepackage{hyperref}

% For inserting graphics and images
\usepackage{graphicx}
\usepackage{float}
\usepackage[footnotesize]{caption}



%%%%%%%%%%%%%%%%%%%%%%%%%%%%%%%%%%%%%%%%%%%%%%%%%%%%%%%%%%%
%                                                         %
%                   USER-DEFINED COMMANDS                 %
%                                                         %
%%%%%%%%%%%%%%%%%%%%%%%%%%%%%%%%%%%%%%%%%%%%%%%%%%%%%%%%%%%

% Make a hyperlink with colored text
\newcommand{\hrefcolor}[3]{\href{#1}{\textcolor{#3}{#2}}}

% Make a hyperlink with gray text
\newcommand{\hrefgray}[2]{\hrefcolor{#1}{#2}{Gray}}

% Make the header for the first page
\title{\textbook\\\normalsize\textsl{by} \textsc{\textauthor}}
\author{\small\textsl{Solutions by}\\\large\myname}
\date{\thedate}

% Make problem list for "title" of page
\newcommand{\problemlist}{ 
\begin{center}
\textsf{\Large \assignment}\\
\textit{\textsf{\problemset}}
\end{center}
\bigskip
}

%~~~~~~~~~~~~~~~~~~~~~~~~~~~~~~~~~~~~~~~~~~~~~~~~~~~~~~~~~%
%                                                         %
%               LETTERS, FUNCTIONS, AND TEXT              %
%                                                         %
%~~~~~~~~~~~~~~~~~~~~~~~~~~~~~~~~~~~~~~~~~~~~~~~~~~~~~~~~~%

% A
\newcommand{\cA}{\mathcal{A}}
\newcommand{\sA}{\mathscr{A}}
\renewcommand{\aa}{\;\text{a.a.}}
\renewcommand{\ae}{\;\text{a.e.}}
% B
\newcommand{\B}{\mathscr{B}}
\newcommand{\cB}{\mathcal{B}}
% C
\newcommand{\cC}{\mathcal{C}}
\newcommand{\cov}{\text{cov}}
% E
\newcommand{\E}{\mathbb{E}}
% F
\newcommand{\sF}{\mathscr{F}}
\newcommand{\cF}{\mathcal{F}}
\newcommand{\Ft}{F^\sim}
% G
\newcommand{\cG}{\mathcal{G}}
\newcommand{\sG}{\mathscr{G}}
% I
\newcommand{\io}{\;\text{i.o.}}
% N
\newcommand{\N}{\mathbb{N}}
% P
\newcommand{\cP}{\mathcal{P}}
\newcommand{\sP}{\mathscr{P}}
\newcommand{\pr}{\text{pr}}
% Q
\newcommand{\Q}{\mathbb{Q}}
% R
\newcommand{\R}{\mathbb{R}}
\newcommand{\bR}{\mathbf{R}}
\newcommand{\cR}{\mathcal{R}}
% S
\newcommand{\cS}{\mathcal{S}}
% U
\newcommand{\cU}{\mathcal{U}}
% V
\newcommand{\var}{\text{var}}
% Z
\newcommand{\Z}{\mathbb{Z}}
% Punctuation
\newcommand{\sbs}{\;|\;} % space bar space
% Math
\newcommand{\imii}{\int_{-\infty}^\infty}
\newcommand{\pion}{\prod_{i=1}^n}
\newcommand{\pioI}{\prod_{i=1}^I}
\newcommand{\pjon}{\prod_{j=1}^n}
\newcommand{\pjoJ}{\prod_{j=1}^J}
\newcommand{\pkon}{\prod_{k=1}^n}
\newcommand{\pkoK}{\prod_{k=1}^K}
\newcommand{\sion}{\sum_{i=1}^n}
\newcommand{\sioI}{\sum_{i=1}^I}
\newcommand{\sjon}{\sum_{j=1}^n}
\newcommand{\sjoJ}{\sum_{j=1}^J}
\newcommand{\skon}{\sum_{k=1}^n}
\newcommand{\skoK}{\sum_{k=1}^K}
\newcommand{\sioi}{\sum_{i=1}^\infty}
\newcommand{\sjoi}{\sum_{j=1}^\infty}
\newcommand{\skoi}{\sum_{k=1}^\infty}
\newcommand{\sio}{\sum_{i=1}}
\newcommand{\sjo}{\sum_{j=1}}
\newcommand{\sko}{\sum_{k=1}}
% Typography
\newcommand{\scb}[1]{\textsc{\textbf{#1}}}

%~~~~~~~~~~~~~~~~~~~~~~~~~~~~~~~~~~~~~~~~~~~~~~~~~~~~~~~~~%
%                                                         %
%            CHANGE THESE BASED ON THE PAPER              %
%                                                         %
%~~~~~~~~~~~~~~~~~~~~~~~~~~~~~~~~~~~~~~~~~~~~~~~~~~~~~~~~~%

% Constants for fancy header and first page info
\newcommand{\mynamehdr}{\hrefgray{http://biostat.jhsph.edu/~jpringle/}{\myname}}
\newcommand{\textbook}{Probability and Measure}
\newcommand{\myname}{James K. Pringle}
\newcommand{\textauthor}{Billingsley}
\newcommand{\assignment}{Section 1}
\newcommand{\thedate}{\today}
\newcommand{\problemset}{Problems 1, 2, 3, 4, 5, 6, 7}

%%%%%%%%%%%%%%%%%%%%%%%%%%%%%%%%%%%%%%%%%%%%%%%%%%%%%%%%%%%
%                                                         %
%                      BEGIN DOCUMENT                     %
%                                                         %
%%%%%%%%%%%%%%%%%%%%%%%%%%%%%%%%%%%%%%%%%%%%%%%%%%%%%%%%%%%
\begin{document}

% Add title
\maketitle

\bigskip

% Add problem set description
\problemlist

\begin{enumerate}
%%%%%%%%%%%%%%%%%%%%%%%%%%%%%%%%%%%%%%%%%%%%%%%%%%%%%%%%%%%
%                                                         %
%                     Start Problem 1                     %
%                                                         %
%%%%%%%%%%%%%%%%%%%%%%%%%%%%%%%%%%%%%%%%%%%%%%%%%%%%%%%%%%%
\item[1.1]
\begin{enumerate}
\item[(a)]
Show that a \textit{discrete} probability space (see Example 2.8 for the formal definition) cannot contain an infinite sequence $A_1, A_2, \dots$ of independent events each of probability $\frac{1}{2}$. Since $A_n$ could be identified with heads on the $n$th toss of a coin, the existence of such a sequence would make this section superfluous.

\begin{proof}
Apply 1.1(b). Let $p_n = 0.5$. Therefore $\alpha_n=\min\{0.5,0.5\} = 0.5$ and $\sum_n \alpha_n$ diverges. Hence a discrete probability space cannot contain independent events $A_1, A_2, \dots$ each with probability $\frac{1}{2}$. 
\end{proof}

\item[(b)]
Suppse that $0 \leq p_n \leq 1$, and put $\alpha_n = \min\{p_n, 1-p_n\}$. Show that if $\sum_n \alpha_n$ diverges, then no discrete probability space can contain independent events $A_1, A_2, \dots$ such that $A_n$ has probability $p_n$.

\begin{proof}
Suppose $B_i$ is $A_i$ or $A_i^c$. Note $\alpha_n \leq P(B_n) \leq 1-\alpha_n$. It follows that
\begin{align}
0 \leq P(B_1 \cap \cdots \cap B_n) &= P(B_1)\cdots P(B_n) \\
&\leq \pion (1 -\alpha_i) \\
&\leq \exp \left[ -\sion \alpha_i \right] 
\label{ineq1}
\end{align}
Inequality \eqref{ineq1} comes from combining several $1+x \leq e^x$ with the added condition that $0 \leq 1+x$. Since $\sum_n \alpha_n$ diverges, $\exp \{ -\sion \alpha_i \} \downarrow 0$.  Taking the limit in $n$ of the inequality above, 
\begin{equation}
0 \leq P\left( \cap_{i=1}^\infty B_i  \right) \leq 0.
\end{equation}

For all $i$, each $\omega$ in the sample space is in $A_i$ or $A_i^c$. Therefore, for all $n$, each $\omega$ is in $\cap_{i=1}^n B_i$. Thus, $\omega \in \cap_{i=1}^\infty B_i$ and
\begin{equation}
P(\omega) \leq P(\cap_{i=1}^\infty B_i) = 0
\end{equation}
Supposing that $\Omega$ were a discrete space leads to the conclusion that
\begin{equation}
1 = P(\Omega) = \sum_{\omega \in \Omega} P(\omega) = \sum_{\omega \in \Omega} 0 = 0,
\end{equation}
a contradiction. Thus no discrete probability space can contain independent events $A_1, A_2, \dots$ such that $A_n$ has probability $p_n$.

\end{proof}

\end{enumerate}

\item[1.2]
Show that $N$ and $N^c$ are dense [A15] in (0,1].

\begin{proof}
The definition of \textit{dense}: The set $A$ is \textit{dense} in the set $B$ if for each $x$ in $B$ and each open interval $J$ containing $x$, $J$ meets $A$ ($J \cap A \neq \emptyset$).

\textbf{Part 1, $N$ is dense:} From Theorem 1.2, $N$ has negligible complement, and therefore $P(N)=1$. Suppose $N$ is not dense. Then there exists an open interval $J = (a,b) \in (0,1]$ (with $a \neq b$) such that $J \cap N = \emptyset$. Since this section discusses probabilities of intervals open on the left and closed on the right, take $J' = (a, \frac{a+b}{2}] \subset J$. It follows that $J' \cap N = \emptyset$. Therefore, since $N \subset (0,1] \setminus J' = (0,a] \cup (\frac{a+b}{2}, 1]$,
\begin{align}
P(N) \leq P((0,1] \setminus J') 
&= P\left((0,a] \cup \left(\frac{a+b}{2}, 1\right]\right) 
\\
&= P((0,a]) + P\left(\left(\frac{a+b}{2}, 1\right]\right) 
\\
&= a + \left(1 - \frac{a+b}{2}\right) 
\\
&= 1 + \frac{a}{2} - \frac{b}{2} 
\\
&< 1,
\end{align}
a contradiction, since $P(N) = 1$. Therefore $N$ is dense.

\textbf{Part 2, $N^c$ is dense:} Let $J$ be a non-trivial open interval $(a,b)$ in $(0,1]$. Let $K = \min \{k : 2^{-k} < (b-a)/2, \; k \in \Z^+ \}$. By definition, $K$ is constructed such that the length of a dyadic interval of rank $K$ is the largest dyadic interval length less than $\frac{b-a}{2}$. This choice of $K$ guarantees that at least one of the $2^K$ dyadic intervals of rank $K$ that decompose the unit interval is contained in $J$. In other words for some integer $n$ with $0 \leq n \leq 2^{K} - 1$, it follows that $I = (n / 2^K, (n+1)/2^K] \subset J$. Note, $\omega \in I$ means that $d_i(\omega)$ is fixed for $i \leq K$. Next, consider a finite sequence, $S$, of $0$'s and $1$'s---not all $0$'s---the mean of which is $\mu \neq \frac{1}{2}$. For the purposes of this proof, let $S_1 = 1, S_2=1, S_3=0$. Choose the $\omega \in I$ such that $d_{K+n}(\omega) = S_{K+n \mod |S|} = S_{K+n \mod 3}$ for $n \in \Z^+$. Calculating,
\begin{align}
\label{ineq2}
\lim_{n \to \infty} \frac{1}{n} \sion d_i(\omega)
&= 
\lim_{n \to \infty} \frac{1}{n|S|} \sio^{n|S|} d_i(\omega) \\
&= 
\lim_{n \to \infty} \frac{1}{n|S|} 
\left(\sio^K d_i(\omega) + \sum_{i = K+1}^{n|S|} d_i(\omega) \right) \\
&=\lim_{n \to \infty} \frac{1}{n|S|} 
\sio^K d_i(\omega)
+ \lim_{n \to \infty} \frac{1}{n|S|} 
\sum_{i = K+1}^{n|S|} d_i(\omega) \\
&= 0 + \mu \\
& \neq \frac{1}{2}
\label{neq1}
\end{align}
Since the series of partial sums converges to a limit on the left-hand side, all subsequences converge to that same limit, thus justifying \eqref{ineq2}. By construction, $\omega$ is in $J$, and by \eqref{neq1}, $\omega \in N^c$ and thus $N^c$ is dense in $(0,1]$.
\end{proof}

\item[1.3]
$\uparrow$ Define a set $A$ to be \textit{trifling} if for each $\epsilon$ there exists a \textit{finite} sequence of intervals $I_k$ satisfying (1.22) and (1.23). This definition and the definition of negligibility apply as they stand to all sets on the real line, not just to subsets of $(0,1]$.
\begin{enumerate}[(a)]
\item
Show that a trifling set is negligible
\begin{proof}
Definition of \textbf{negligible}: A subset $A$ of $\Omega$ is negligible if for each positive $\epsilon$ there exists a finite or countable collection $I_1, I_2, \dots$ of intervals (they may overlap) satisfying
\begin{equation}
A \subset \bigcup_k I_k
\end{equation}
and 
\begin{equation}
\sum_k |I_k| < \epsilon
\end{equation}
For trifling set $A$, take the finite sequence of intervals $I_k$ and use them for the finite or countable collection of intervals for the definition of negligible. Therefore $A$ is also negligible.
\end{proof}
\item
Show that the closure of a trifling set is also trifling.
\begin{proof}
Suppose $A$ is trifling and let $A^-$ be its closure. Given $\epsilon > 0$, choose intervals $(a_k, b_k], k=1,\dots,n$, such that $A \subset \cup_{k=1}^n (a_k, b_k]$ and $\sum_{k=1}^n (b_k - a_k) < \epsilon/2$. Let $x_k = a_k - \epsilon/2$. Then $A^- \subset \cup_{k=1}^n (x_k, b_k]$ and $\sum_{k=1}^n (b_k - x_k) < \epsilon$.
\end{proof}
\item
Find a bounded negligible set that is not trifling.
\begin{proof}

\end{proof}
\item
Show that the closure of a neglibigle set may not be negligible.
\item
Show that finite unions of trifling sets are trifling but that this can fail for countable unions.
\end{enumerate}

\item[1.4]
$\uparrow$ For $i = 0, \dots, r-1$, let $A_r(i)$ be the set of numbers in $(0,1]$ whose nonterminating expansions in the base $r$ do not contain digit $i$.
\begin{enumerate}[(a)]
\item
Show that $A_r(i)$ is trifling.
\item
Find a trifling set $A$ such that every point in the unit interval can be represented in the form $x+y$ with $x$ and $y$ in $A$.
\item
Let $A_r(i_l, \dots, i_k)$ consist of the numbers in the unit interval in whose base-$r$ expansion the digits $i_l, \dots, i_k$ nowhere appear consecutively in that order. Show that it is trifling. What does this imply about the monkey that types at random?
\end{enumerate}

\item[1.5]
$\uparrow$ The \textit{Cantor set} $C$ can be defined as the closure of $A_3(1)$.
\begin{enumerate}[(a)]
\item
Show that $C$ is uncountable but trifling.
\item
From $[0,1]$ remove the open middle third $\left(\frac{1}{3}, \frac{2}{3}\right)$; from the remainder, a union of two closed intervals, remove the two open middle thirds $\left(\frac{1}{9},\frac{2}{9}\right)$ and $\left(\frac{7}{9},\frac{8}{9}\right)$. Show that $C$ is what remains when this process is continued ad infinitum.
\item
Show that $C$ is perfect [A15].
\end{enumerate}

\item[1.6]
Put $M(t) = \int_0^1 e^{ts_n(\omega)} d\omega$, and show by successive differentiation under the integral that
\begin{equation}
M^{(k)}(0) = \int_0^1 s_n^k(\omega) d\omega.
\end{equation}
Over each dyadic interval of rand $n$, $s_n(\omega)$ has a constant value of the form $\pm 1 \pm 1 \pm \cdots \pm 1$, and therefore $M(t) = 2^{-n} \sum \exp t(\pm 1 \pm 1 \pm \cdots \pm 1)$, where the sum extends over all $2^n$ $n$-long sequences of $+1$'s and $-1$'s. Thus
\begin{equation}
M(t) = \left(  \frac{e^t+e^{-t}}{2}   \right) ^n = (\cosh t)^n.
\end{equation}
Use this and (1.38) to give new proofs of (1.16), (1.18), and (1.28). (This, the moethod of moment generating functions, will be investigated systematically in Section 9).

\item[1.7]
$\uparrow$ By an argument similar to that leading to (1.39) show that the Rademacher functions satisfy
\begin{align}
\int_0^1 \exp \left[ i \sum_{k=1}^n a_k r_k(\omega) \right] d\omega
&=
\prod_{k=1}^n \frac{e^{ia_k} + e^{-ia_k}}{2}
\\
&= \prod_{k=1}^n \cos a_k.
\end{align}
Take $a_k = t2^{-k}$, and from $\skoi r_k(\omega) 2^{-k} = 2\omega -1$ deduce
\begin{equation}
\frac{\sin t}{t} = \prod_{k=1}^n \cos \frac{t}{2^k}
\end{equation}
by letting $n \to \infty$ inside the integral above. Derive Vieta's formula
\begin{equation}
\frac{2}{\pi} = \frac{\sqrt{2}}{2} \frac{\sqrt{2 + \sqrt{2}}}{2} 
\frac{\sqrt{2 + \sqrt{2 + \sqrt{2}}}}{2} \cdots.
\end{equation}

\end{enumerate}
\end{document}