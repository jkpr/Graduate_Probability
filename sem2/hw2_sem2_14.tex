% /**
%  * A template for homework files in math classes. The 
%  * packages and newcommands are a good starting point.
%  *
%  * Author: James K. Pringle
%  * E-mail: jameskpringle@gmail.com
%  * Last Changed: 26 February 2013
%  *
%  * "LaTeX countains the increasing union of MS Word"
%  */
%~~~~~~~~~~~~~~~~~~~~~~~~~~~~~~~~~~~~~~~~~~~~~~~~~~~~~~~~~%
%%%%%%%%%%%%%%%%%%%%%%%%%%%%%%%%%%%%%%%%%%%%%%%%%%%%%%%%%%%
%                                                         %
%                        PAGE SETUP                       %
%                                                         %
%%%%%%%%%%%%%%%%%%%%%%%%%%%%%%%%%%%%%%%%%%%%%%%%%%%%%%%%%%%
\documentclass[letterpaper, 12pt]{article}

% 1in margins all the way around
\usepackage[margin=1in]{geometry}

% Sets \parindent to 0 and \parskip to stretchable.
\usepackage{parskip}
% Use for bigger spaces between paragraphs.
%\parskip=1.5\baselineskip

% Set headers and footers
\usepackage{fancyhdr}
\pagestyle{fancy}
% Header
\renewcommand{\headrulewidth}{0.4pt}
\lhead{\textsc{\mathclass}}
\chead{\textsc{\today}}
\rhead{\textsc{\mynamehdr}}
% Footer
\renewcommand{\footrulewidth}{0.4pt}
\lfoot{}
\cfoot{\thepage}
\rfoot{}

% Make the space between lines slightly more generous 
% than normal single spacing, but compensate so that the 
% spacing between rows of matrices still looks normal.  
% Note that 1.1=1/.9090909...
\renewcommand{\baselinestretch}{1.1}
\renewcommand{\arraystretch}{.91}

%%%%%%%%%%%%%%%%%%%%%%%%%%%%%%%%%%%%%%%%%%%%%%%%%%%%%%%%%%%
%                                                         %
%                      USEFUL PACKAGES                    %
%                                                         %
%%%%%%%%%%%%%%%%%%%%%%%%%%%%%%%%%%%%%%%%%%%%%%%%%%%%%%%%%%%

% The classic three
\usepackage{amsmath,amsthm,amssymb}

% Define \newtheorem for use
% No numbers, labeled 'Theorem'
\newtheorem*{nthm}{Theorem}

% Not sure what this is for
\usepackage{amsfonts}

% Fancy script font
\usepackage{mathrsfs}

% Makes enumerate environment much easier to customize
% by specifying the counter
\usepackage{enumerate}

% Color
\usepackage{color}
\usepackage[usenames,dvipsnames,svgnames,table]{xcolor}

% URL links
\usepackage{hyperref}

%%%%%%%%%%%%%%%%%%%%%%%%%%%%%%%%%%%%%%%%%%%%%%%%%%%%%%%%%%%
%                                                         %
%                   USER-DEFINED COMMANDS                 %
%                                                         %
%%%%%%%%%%%%%%%%%%%%%%%%%%%%%%%%%%%%%%%%%%%%%%%%%%%%%%%%%%%

% Make a hyperlink with colored text
\newcommand{\hrefcolor}[3]{\href{#1}{\textcolor{#3}{#2}}}

% Make a hyperlink with gray text
\newcommand{\hrefgray}[2]{\hrefcolor{#1}{#2}{Gray}}

% Make the header for the first page
\newcommand{\firstpageinfo}{
\textsf{
\begin{flushleft}
\sc \myname \\
\normalfont \mathclass \\
\professorname \\
\assignmentnumber \\
\thedate
\end{flushleft}
} \bigskip
}

% Make problem list for "title" of page
\newcommand{\problemlist}{ 
\begin{center}
\textbf{\Large \textsf{\assignmentnumber}}\\
\textit{\textsf{\problemset}}
\end{center}
}

%~~~~~~~~~~~~~~~~~~~~~~~~~~~~~~~~~~~~~~~~~~~~~~~~~~~~~~~~~%
%                                                         %
%               LETTERS, FUNCTIONS, AND TEXT              %
%                                                         %
%~~~~~~~~~~~~~~~~~~~~~~~~~~~~~~~~~~~~~~~~~~~~~~~~~~~~~~~~~%

% A
\newcommand{\cA}{\mathcal{A}}
\newcommand{\sA}{\mathscr{A}}
\renewcommand{\aa}{\;\text{a.a.}}
\renewcommand{\ae}{\;\text{a.e.}}
% B
\newcommand{\B}{\mathscr{B}}
\newcommand{\cB}{\mathcal{B}}
% C
\newcommand{\cC}{\mathcal{C}}
% E
\newcommand{\E}{\mathbb{E}}
% F
\newcommand{\sF}{\mathscr{F}}
\newcommand{\cF}{\mathcal{F}}
\newcommand{\Ft}{F^\sim}
% G
\newcommand{\cG}{\mathcal{G}}
\newcommand{\sG}{\mathscr{G}}
% I
\newcommand{\io}{\;\text{i.o.}}
% N
\newcommand{\N}{\mathbb{N}}
% P
\newcommand{\cP}{\mathcal{P}}
\newcommand{\sP}{\mathscr{P}}
% Q
\newcommand{\Q}{\mathbb{Q}}
\newcommand{\cQ}{\mathcal{Q}}
\newcommand{\fQ}{\mathfrak{Q}}
\newcommand{\sQ}{\mathscr{Q}}
% R
\newcommand{\R}{\mathbf{R}}
\newcommand{\cR}{\mathcal{R}}
% S
\newcommand{\cS}{\mathcal{S}}
% U
\newcommand{\cU}{\mathcal{U}}
\newcommand{\fU}{\mathfrak{U}}
\newcommand{\sU}{\mathscr{U}}
% V
\newcommand{\var}{\text{var}}
% Z
\newcommand{\Z}{\mathbb{Z}}
\newcommand{\cZ}{\mathcal{Z}}

%~~~~~~~~~~~~~~~~~~~~~~~~~~~~~~~~~~~~~~~~~~~~~~~~~~~~~~~~~%
%                                                         %
%            CHANGE THESE BASED ON THE PAPER              %
%                                                         %
%~~~~~~~~~~~~~~~~~~~~~~~~~~~~~~~~~~~~~~~~~~~~~~~~~~~~~~~~~%

% Constants for fancy header and first page info
\newcommand{\mynamehdr}{\hrefgray{http://biostat.jhsph.edu/~jpringle/}{\myname}}
\newcommand{\mathclass}{550.621 Probability}
\newcommand{\myname}{James K. Pringle}
\newcommand{\professorname}{Dr. Jim Fill}
\newcommand{\assignmentnumber}{Assignment 2}
\newcommand{\thedate}{\today}
\newcommand{\problemset}{All the exercises on the transition probabilities handout}

%%%%%%%%%%%%%%%%%%%%%%%%%%%%%%%%%%%%%%%%%%%%%%%%%%%%%%%%%%%
%                                                         %
%                      BEGIN DOCUMENT                     %
%                                                         %
%%%%%%%%%%%%%%%%%%%%%%%%%%%%%%%%%%%%%%%%%%%%%%%%%%%%%%%%%%%
\begin{document}

% Take header off of first page
\thispagestyle{empty}

% Put in first page info (top of page)
\firstpageinfo

% Put in title for the paper
\problemlist

\begin{enumerate}
%%%%%%%%%%%%%%%%%%%%%%%%%%%%%%%%%%%%%%%%%%%%%%%%%%%%%%%%%%%
%                                                         %
%                     Start Problem 1                     %
%                                                         %
%%%%%%%%%%%%%%%%%%%%%%%%%%%%%%%%%%%%%%%%%%%%%%%%%%%%%%%%%%%
\item
Let $(\Omega_1, \cA_1)$ and $(\Omega_2, \cA_2)$ be two measurable spaces and let $\pi_i : \Omega_1 \times \Omega_2 \rightarrow \Omega_i$ be the $i$th projection, $i = 1,2$. Set 
\begin{align*}
\cC &:= \pi_1^{-1}(\cA_1) \cup \pi_2^{-1}(\cA_2) 
\text{, the class of \textit{measurable cylinders,}} \\
\cR &:= \{   A_1 \times A_2 : A_1 \in \cA_1, A_2 \in \cA_2  \}
\text{, the class of \textit{measurable rectangles,}} \\
\cU &:= \left\lbrace \sum _{j\in J}  R_j : J \text{ finite, } R_j \in  \cR  \text{ for each }j \right\rbrace \text{.}
\end{align*}
\begin{enumerate}[(a)]
\item
Show $\cC$ is closed under complementation.
\begin{proof}
Given an element $C$ of $\cC$, we have  $C = A_1 \times \Omega_2$ or $C = \Omega_1 \times A_2$ for some $A_1 \in \cA_1$ or $A_2 \in \cA_2$. 
(As an aside, note since $\Omega_1 \in \cA_1$ and $\Omega_2 \in \cA_2$ by definition of $\sigma$-field, then $C \in \cR$. Hence $\cC \subset \cR$). 
The complement of $C$ is  $C^c = A_1^c \times \Omega_2$ or $C^c = \Omega_1 \times A_2^c$. 
Since $\sigma$-fields are closed under complementation, $A_1^c \in \cA_1$ and $A_2^c \in \cA_2$. 
Therefore $C^c \in \cC$.
\end{proof}
\item
Show $\cR$ is a $\pi$-system.
\begin{proof}
Let $A_1 \times A_2 \in \cR$ and $B_1 \times B_2 \in \cR$ where $A_i$ and $B_i$ are elements of $\cA_i$ for $i=1,2$.  
Taking the intersection of both sets, we have
\begin{equation*}
(A_1 \times A_2) \cap (B_1 \times B_2) = (A_1 \cap B_1) \times (A_2 \cap B_2)
\end{equation*}
Since $A_i$ and $B_i$ are elements of $\cA_i$ for $i=1,2$ it follows that $A_i \cap B_i \in \cA_i$ for $i = 1,2$ by closure under countable intersections of $\sigma$-fields. Thus $(A_1 \times A_2) \cap (B_1 \times B_2) = (A_1 \cap B_1) \times (A_2 \cap B_2) \in \cR$. Hence $\cR$ is a $\pi$-system.
\end{proof}
\item
Show $\cU$ is the field generated by $\cC$ (and by $\cR$).
\begin{proof}
First, we show that $\cU$ is a field.

\textbf{Closure under binary intersection:}
Suppose $U_1 \in \cU$ and $U_2 \in \cU$. 
Then $U_1 = \sum_{j=1}^J R_j$ and let $U_2 = \sum_{k=1}^K S_k$ where $R_j \in \cR$ and $S_k \in \cR$ for each $j$ and $k$ and $J$ and $K$ are finite.
\begin{align*}
U_1 \cap U_2 
&=
\sum_{j=1}^J R_j
\bigcap
\sum_{k=1}^K S_k
\\
&=
\cup_{j=1}^J R_j
\bigcap
\cup_{k=1}^K S_k
\\
&=
\cup_{j=1}^J 
\left(
R_j 
\bigcap \cup_{k=1}^K S_k
\right)
\\
&=
\cup_{j=1}^J 
\left(
\cup_{k=1}^K
R_j 
\cap S_k
\right)
\end{align*}
For each $j$ and $k$, since $\cR$ is a $\pi$-system,
$R_j \cap S_k = T_{jk}$ is a rectangle. Furthermore,
suppose $(i,j) \neq (i',j')$. Then
\[
T_{jk} \cap T_{j'k'}=
R_j \cap S_k
\bigcap
R_{j'} \cap S_{k'}
=
R_j \cap R_{j'}
\bigcap
S_{k} \cap S_{k'}
= \emptyset
\] 
because $\{R_j\}$ are disjoint and $\{S_k\}$ are disjoint.
Therefore, 
\[
U_1 \cap U_2 = 
\cup_{j=1}^J 
\left(
\cup_{k=1}^K
R_j 
\cap S_k
\right)
=
\sum_{\substack{j \in \{1, \cdots, J\} \\ k \in \{1, \cdots, K\}}}
T_{jk}
\in 
\cU
\]
Thus $\cU$ is closed under binary intersection. 

Next we show closure under intersection of finite terms. The base case is already proven above ($K=2$). Suppose for some integer $K > 2$
\[
\cap_{k=1}^K U_k \in \cU
\]
where $U_k \in \cU$ for each $k$. Then suppose $U_1, \cdots, U_{K+1}$ are members of $\cU$. Calculating,
\[
\cap_{k=1}^{K+1} U_k 
= 
\left(
\cap_{k=1}^{K} U_k 
\right) \cap U_{K+1}
\]
is the intersection of two elements of $\cU$ by inductive hypothesis. Then by the base case their intersection is in
$\cU$. Hence
\[
\cap_{k=1}^{K+1} U_k  \in \cU
\]
By induction, the intersection of any $k$ elements of $\cU$ is a member of $\cU$ for all positive integers $k$.

\textbf{Closure under complementation:}
First a little lemma. 
Let $R = A_1 \times A_2$ be a rectangle in $\cR$. 
Then 
\[
R^c = (A_1^c \times A_2) \cup (A_1 \times A^c_2) \cup (A_1^c \times A_2^c)
\]
is a finite union of disjoint rectangles in $\cR$. Hence $R^c \in \cU$.

Suppose $U = \sum_{j=1}^J R_j \in \cU$ where $J$ is finite and $R_j \in \cR$ for each $j$. Then
\begin{align*}
U^c &= 
\left(
\sum_{j=1}^J R_j
\right)^c
\\
&=
\left(
\cup_{j=1}^J R_j
\right)^c
\\
&=
\cap_{j=1}^J R_j^c
\end{align*}
Since $R_j^c \in \cU$ by the little lemma and intersection of a finite number of members of $\cU$ lies in $\cU$ (as proved above with induction), it follows that
\[
U^c = \cap_{j=1}^J R_j^c \in \cU
\]  
Since closure of binary intersection is equivalent to closure of binary union under closure of complementation (Chung 17), it is demonstrated that $\cU$ is a field.

\textbf{$\cU$ is the field generated by $\cR$ and $\cC$:}
It was shown in (a) that $\cC \subset \cR$. Since $\langle \cR \rangle$ contains $\cR$ 
(the angle brackets denote ``field generated by''), it follows that 
\begin{equation}
\label{crr}
\cC \subset \cR \subset \langle \cR \rangle
\end{equation}
Consider $\langle \cC \rangle$. Let $R \in \cR$. Then
$R = A_1 \times A_2$ for $A_1 \in \cA_1$ and $A_2 \in \cA_2$. Then since $A_1 \times \Omega_2 \in \cC$ and $\Omega_1 \times A_2 \in \cC$ and since $\langle \cC \rangle$ is closed under binary intersection, 
\[
R = A_1 \times A_2 = (A_1 \times \Omega_2) \cap (\Omega_1 \times A_2) \in \langle \cC \rangle
\]
Therefore $\cR \subset \langle\cC\rangle$. By definition of ``field generated by,'' the field generated by a set is a subset of all other fields that contain that set. Thus
\[
\cR \subset \langle\cR\rangle \subset \langle\cC\rangle
\qquad
\text{and}
\qquad
\cC \subset \langle \cC \rangle \subset \langle\cR\rangle
\text{ by \eqref{crr}}
\] 
Therefore $\langle \cC \rangle  = \langle \cR \rangle$.


Let $R \in \cR$, it is the case that $R$ is the sum of one rectangle (itself). Thus $R \in \cU$, and $\cR \subset \cU$. Since $\cU$ is a field containing $\cR$, by the same reasoning above,
\begin{equation}
\label{rru}
\cR 
\subset 
\langle \cR \rangle
\subset
\cU
\end{equation}
Let $U = \sum_{j=1}^J R_j \in \cU$ where $J$ is finite and $R_j \in \cR$ for each $j$. Then $U = \cup_{j=1}^J R_j$. Since closure of binary union implies closure of finite union (see Chung 17), then $\langle \cR \rangle$ is closed under finite union. Hence 
\[
U = \sum_{j=1}^J R_j = \cup_{j=1}^J R_j \in \langle \cR \rangle
\]
Therefore $\cU \subset \langle \cR \rangle$. Combining that with \eqref{rru}, there is set equality, i.e.
\[
\cU = \langle \cR \rangle
\]
Thus 
\[
\cU = \langle \cR \rangle = \langle \cC \rangle
\]
since $\langle \cR \rangle = \langle \cC \rangle$ as shown above.
\end{proof}

\end{enumerate}
%We already have $\cC \subset \cR \subset \cU$.
%Let $\langle \cC \rangle$ denote the field generated by $\cC$.
%Let $U \in \cU$. 
%By definition, $U = \sum R_j$ (a finite sum).
%Each $R_i$ in the sum can be written as $A_{i1} \times A_{i2}$. 
%By \eqref{eq1}, $A_{i1} \times A_{i2} = (A_{i1} \times \Omega_2) \cap (\Omega_1 \times A_{i2})$, the intersection of elements of $\cC$. 
%Since $\langle \cC \rangle$ is generated from finite complementation, union, and intersection of sets in $\cC$, it is clear that each $R_i \in \langle \cC \rangle$ (by finite intersection). 
%By this same reasoning, every rectangle is in $\langle \cC \rangle$. 
%Thus $\langle \cR \rangle \subset \langle \cC \rangle$.
%Furthermore, $U = \sum R_j \in \langle \cR \rangle \subset \langle \cC \rangle$ by finite union. 
%This shows $\cU \subset \langle \cR \rangle \subset \langle \cC \rangle$. 
%
%Now we show $\cU$ is field. 
%Once we have shown that, it will be clear that the smallest field containing $\cC$, which is $\langle \cC \rangle$, is a subset of $\cU$. 
%That, combined with $\cU \subset \langle \cC \rangle$, will show $\cU = \langle \cC \rangle$. 
%Likewise, that will show that the smallest field containing $\cR$, which is $\langle \cR \rangle$, is a subset of $\cU$. 
%That combined with $\cU \subset \langle \cR \rangle$ will give $\cU =\langle \cR \rangle$.
%
%\textbf{Closure under binary union:} Let $U_1, U_2 \in \cU$ with $U_i = \sum_j^{J_i} R_{ij}$, a finite sum over $j$ with $i = 1,2$. 
%Notice elements of $\cU$ are finite unions of disjoint rectangles. 
%We now show that the union of $U_1$ and $U_2$ is a union of disjoint rectangles. 
%Define 
%\begin{equation}
%\label{eq2}
%B_{1k} = R_{1k} \setminus (\sum_j R_{2j}) = R_{1k} \setminus R_{21} \setminus R_{22} \setminus \cdots \setminus R_{2J_2}
%\end{equation}
%where here we are evaluating the binary set difference operators from left to right (written that way to avoid copious amounts of  parentheses), and $k \in \{1, 2, \cdots, K\}$. 
%Since 
%\[
%(A_1 \times A_2) \setminus (B_1 \times B_2) = (A_1 \cap B_1^c) \times (A_2 \times B_2^c) + (A_1 \cap B_1) \times (A_2 \times B_2^c) + (A_1 \cap B_1^c) \times (A_2 \times B_2)
%\text{,}
%\]
%a sum of rectangles, we have by "quick" induction (quick because the obvious inductive step---a sum of rectangles minus a last rectangle is the sum of each rectangle in the sum minus the last rectangle---is skipped) that \eqref{eq2} is a sum of rectangles. 
%Therefore, each $B_{1j}$ is a sum of rectangles. 
%Since $B_{1j} \subset R_{1j}$ the $B_{1j}$ are mutually disjoint. 
%Then by construction, $B_{1j}$ and $R_{2j}$ are mutually disjoint and their sum is the union is the finite sum of disjoint rectangles equal to the union of $U_1$ and $U_2$. 
%Hence $\cU$ is closed under binary union.
%
%\textbf{Closure under complementation:} First a little lemma. 
%Let $R = A_1 \times A_2$ be a rectangle in $\cR$. 
%Then 
%\[
%R^c = (A_1^c \times A_2) \cup (A_1 \times A^c_2) \cup (A_1^c \times A_2^c)
%\]
%is a finite union of rectangles in $\cR$. 
%Let $U \in \cU$. 
%Then $U = \sum_{j=1}^J R_j$ for rectangles $R_j$. 
%We have
%\[
%U^c = \left(\sum_{j=1}^J R_j\right)^c = \bigcap_{j=1}^JR_j^c 
%= \bigcap_{j=1}^J (\cup_{i=1}^{I_j} S_i)
%\]
%where $S_i$ are rectangles by the little lemma. 
%By the distributive law for sets, $\cap_{j=1}^J (\cup_{i=1}^{I_j} S_i)$ is the union of intersections of rectangles (it is hard to write a closed form since $I_j$ is variable).
%By ``quick'' induction, the intersection of any finite number of rectangles is a rectangle (the base case is \eqref{eq1}, and we skip the obvious inductive step). 
%Therefore, the union of intersections of rectangles is the union of rectangles, and we have
%\[
%U^c = \bigcap_{j=1}^J (\cup_{i=1}^{I_j} S_i) = \bigcup_{i = 1}^I T_i
%\]
%where $T_i$ are rectangles and $I$ is finite. 
%Now we show $U^c$ can be reduced to the union of disjoint rectangles.
% We do the classic trick where we let $B_1 = T_1$, then $B_i = T_i \setminus (\cup_{j=1}^{i-1}T_j)$. 
%Then we have $\cup_{i=1}^I B_i = \cup_{i=1}^I T_i$ and the $B_i$ are mutually disjoint. 
%Furthermore, by the ``quick'' induction based on \eqref{eq2}, each $B_i$ is a sum of dijoint rectangles. 
%This shows $U^c \in \cU$ and we conclude $\cU$ is a field. 
%From our argument above, we can finally conclude $\cU = \langle \cC \rangle$ and $\cU = \langle \cR \rangle$. 
%%%%%%%%%%%%%%%%%%%%%%%%%%%%%%%%%%%%%%%%%%%%%%%%%%%%%%%%%%%
%                                                         %
%                     Start Problem 2                     %
%                                                         %
%%%%%%%%%%%%%%%%%%%%%%%%%%%%%%%%%%%%%%%%%%%%%%%%%%%%%%%%%%%
\item
Let $\Omega_1$ and $\Omega_2$ be two spaces; set $\Omega = \Omega_1\times\Omega_2$. 
Let $X:\Omega\to\Psi$ (respectively, $A\subset\Omega$). 
The section of $X$ (resp., of $A$) at $\omega_1\in\Omega_1$ is defined to be the function $X_{\omega_1}:\Omega_2\to\Psi$ 
(resp., the set $A_{\omega_1}\subset\Omega_2$) 
given by $X_{\omega_1}(\omega_2)=X(\omega_1,\omega_2)$ 
(resp., by $A_{\omega_1}=\{ \omega_2\in\Omega_2: (\omega_1,\omega_2)\in A \}$).

Show that $(I_A)_{\omega_1}=I_{A_{\omega_1}}$ for $A\subset\Omega$ and $(X^{-1}(B))_{\omega_1}=X_{\omega_1}^{-1}(B)$ for $B\subset\Psi$.

\begin{proof}
Let $A \subset \Omega$.
Suppose $\omega_2 \in A_{\omega_1}$. 
Thus $(\omega_1, \omega_2) \in A$. 
Therefore, $I_{A_{\omega_1}}(\omega_2) = 1$ 
and $(I_A)_{\omega_1}(\omega_2) = I_A(\omega_1, \omega_2) = 1$.
Both functions agree for $\omega_2 \in A_{\omega_1}$. 
Now suppose $\omega_2 \notin A_{\omega_1}$. 
Then $(\omega_1, \omega_2) \notin A$. 
Thus $I_{A_{\omega_1}}(\omega_2) = 0$ 
and $(I_{A})_{\omega_1} (\omega_2) = I_A (\omega_1, \omega_2) = 0$. 
And we conclude $(I_A)_{\omega_1}=I_{A_{\omega_1}}$ for $A\subset\Omega$.

For the second part, note for all $B \subset \Psi$
\begin{align*}
(X^{-1}(B))_{\omega_1} 
&= \{\omega_2 \in \Omega_2 : (\omega_1, \omega_2) \in X^{-1}(B)\} \\
&= \{\omega_2 \in \Omega_2 : X(\omega_1, \omega_2) \in B\} \\
&= \{\omega_2 \in \Omega_2 : X_{\omega_1}(\omega_2) \in B\} \\
&= (X_{\omega_1})^{-1}(B) \\
&= X_{\omega_1}^{-1}(B)
\end{align*}
as desired.
\end{proof}
%%%%%%%%%%%%%%%%%%%%%%%%%%%%%%%%%%%%%%%%%%%%%%%%%%%%%%%%%%%
%                                                         %
%                     Start Problem 3                     %
%                                                         %
%%%%%%%%%%%%%%%%%%%%%%%%%%%%%%%%%%%%%%%%%%%%%%%%%%%%%%%%%%%
\item
Notations are the same as Problem 2. 
Let $i_{\omega_1}:\Omega_2\to\Omega$ be the injection mapping defined by
\[
i_{\omega_1}(\omega_2) = (\omega_1,\omega_2).
\]
Show that
\[
A_{\omega_1} = i_{\omega_1}^{-1}(A),
\qquad 
X_{\omega_1} = X \circ i_{\omega_1}.
\]

\begin{proof}
For the first part, notice
\begin{align*}
A_{\omega_1} 
&= \{\omega_2 \in \Omega_2 : (\omega_1, \omega_2) \in A \} \\
&= \{\omega_2 \in \Omega_2 : i_{\omega_1}(\omega_2) \in A\} \\
&= (i_{\omega_1})^{-1}(A) \\
&= i_{\omega_1}^{-1}(A)
\end{align*}
as desired. 
Also, for all $\omega_2 \in \Omega_2$ it is the case that 
\[
X_{\omega_1}(\omega_2) = X(\omega_1, \omega_2) = X(i_{\omega_1}(\omega_2)) = X \circ i_{\omega_1} (\omega_2)
\text{.}
\]
Hence $X_{\omega_1} = X \circ i_{\omega_1}$ as desired.
\end{proof}
%%%%%%%%%%%%%%%%%%%%%%%%%%%%%%%%%%%%%%%%%%%%%%%%%%%%%%%%%%%
%                                                         %
%                     Start Problem 4                     %
%                                                         %
%%%%%%%%%%%%%%%%%%%%%%%%%%%%%%%%%%%%%%%%%%%%%%%%%%%%%%%%%%%
\item
Let $\Psi$ be an uncountable set, and let $\cB$ be the $\sigma$-field in $\Psi$ generated by the singletons. 
($\cB$ consists of the countable and co-countable subsets of $\Psi$.) 
Take $(\Omega_1,\cA_1) = (\Psi,\cB) = (\Omega_2,\cA_2)$. 
Consider the diagonal $\Delta:=\{ (\psi,\psi):\psi\in\Psi \}$ of $\cA = \cB\otimes\cB$. 
Show that every section of $\Delta$ is in $\cB$, but $\Delta\notin\cA$.

\begin{proof}
For all $\psi_1 \in \Psi = \Omega_1$, the section 
\[
\Delta_{\psi_1} = \{\psi \in \Psi = \Omega_2 : (\psi_1, \psi) \in \Delta\} = \{\psi_1\}
\]
is an element of $\cB$, since $\cB$ contains the singletons.
Similarly, for all $\psi_2 \in \Psi = \Omega_2$, the section
\[
\Delta_{\psi_2}
= \{\psi \in \Psi = \Omega_1 : (\psi, \psi_2) \in \Delta\} = \{\psi_2\}
\]
is an element of $\cB$. Thus every section of $\Delta$ is in $\cB$.

\textbf{Show $\Delta \notin \cA$:}
Let
\begin{align*}
S_1
&=
\{ \{\psi\} \times \Psi : \psi \in \Psi\}
\\
S_2
&=
\{ \Psi \times \{\psi\} : \psi \in \Psi\}
\\
\cS
&=
S_1 \cup S_2
\end{align*}
Let $\cC$ be the class of cylinders of $\cA$ as defined in problem 1. From the class notes, $\sigma(\cC) = \cA$. We show that $\sigma(\cC) = \sigma(\cS)$.

Let $S \in \cS$. Then $S = \{\psi\} \times \Psi$ or $S = \Psi \times \{\psi\}$ for some $\psi \in \Psi$. 
Since $\cB$ is generated by the singletons, $\{\psi\} \in \cB$. 
Therefore both $\{\psi\} \times \Psi$ and $\Psi \times \{\psi\}$ are cylinders (members of $\cC$).
Furthermore, both $\{\psi\} \times \Psi$ and $\Psi \times \{\psi\}$ are members of $\sigma(\cC)$ since $\cC \subset \sigma(\cC)$.
Since $S$ was arbitrary in $\cS$, 
\[
\cS \subset \cC \subset \sigma(\cC)
\]
Because $\sigma(\cS)$ is a subset of any $\sigma$-field that contains $\cS$, it follows that 
\[
\sigma(\cS) \subset \sigma(\cC).
\]

Define $\cG_1 = \{B : B \times \Psi \in \sigma(\cS)\}$. We show $\cG_1$ is a $\sigma$-field. Since $\sigma(\cS)$ is a $\sigma$-field, $\Psi \times \Psi \in \sigma(\cS)$. Therefore, $\Psi \in \cG_1$. 
Let $B \in \cG_1$. Then $B \times \Psi \in \sigma(\cS)$. Since $\sigma(\cS)$ is a $\sigma$-field,
\begin{align*}
(B \times \Psi)^c 
&= 
(B^c \times \Psi) 
\cup 
(B \times \Psi^c) 
\cup 
(B^c \times \Psi^c) 
\\
&= 
(B^c \times \Psi) 
\cup (B \times \emptyset) 
\cup 
(B^c \times \emptyset)
\\
&=
(B^c \times \Psi)
\cup
\emptyset
\cup
\emptyset
\\
&=
B^c \times \Psi
\end{align*}
is a member of $\sigma(\cS)$. Hence $B^c \in \cG_1$.
Let $B_1, B_2, \cdots$ be a sequence of sets in $\cG_1$. Then
$(B_1 \times \Psi), (B_2 \times \Psi), \cdots$
is a sequence of sets in $\sigma(\cS)$. Thus
\[
\left(
\bigcup_{i=1}^\infty (B_i \times \Psi)
\right)
=
\left(
\bigcup_{i=1}^\infty B_i 
\right)
\times \Psi
\in
\sigma(\cS)
\]
Hence $\cup_{i=1}^\infty B_i \in \cG_1$. Therefore $\cG_1$ is a $\sigma$-field. 
Note for all $\psi\in\Psi$ it is the case that $\{\psi\} \times \Psi \in \sigma(\cS)$ because $\{\psi\} \times \Psi \in S_1$ and $S_1 \subset \cS \subset \sigma(\cS)$. 
It follows that $\cG_1$ contains all the singletons. Since $\cB$ is a subset of all $\sigma$-fields that contain the singletons, we have $\cB \subset \cG_1$.

Similarly, define $\cG_2 = \{B : \Psi \times B \in \sigma(\cS)\}$. 
By similar reasoning to what comes above for $\cG_1$ (substitute $B \times \Psi$ by $\Psi \times B$ and $S_1$ by $S_2$), it follows that $\cB \subset \cG_2$.

Let $C \in \cC$. \textbf{Case 1:} For some $B \in \cB$, we have $C = (B \times \Psi) \in \sigma(\cS)$ because $B \in \cB \subset \cG_1$. \textbf{Case 2:} For some $B \in \cB$, we have $C = (\Psi \times B) \in \sigma(\cS)$ because $B \in \cB \subset \cG_2$. In both cases $C \in \sigma(\cS)$. Since $C$ was arbitrary, $\cC \subset \sigma(\cS)$. Note $\sigma(\cC)$ is a subset of all $\sigma$-fields that contain $\cC$. Hence 
\[
\sigma(\cC) \subset \sigma(\cS)
\]
Since set inclusion is proved in both directions, it follows that $\sigma(\cC) = \sigma(\cS)$. 
%Since $\cB$ is generated by the singletons of $\Psi$, it follows that $\cA = \cB\otimes\cB$ is generated by sets of the form $\{\psi\} \times \Psi$ and $\Psi \times \{\psi\}$ for all $\psi \in \Psi$. 
%Let
%\[
%\cS = \{  \{\psi \} \times \Psi \; | \; \psi \in \Psi \} \cup \{ \Psi \times \{\psi \} \; | \; \psi \in \Psi \}
%\text{,}
%\]
%and with the new notation, $\cA = \sigma(\cS)$.

Note $\cA = \sigma(\cC) = \sigma(\cS)$.
Suppose, by way of contradiction, $\Delta \in \cA$.
Problem 2.9 in Billingsley states
\begin{nthm}
If $B \in \sigma(\sA)$, then there exists a countable subclass $\sA_B$ of $\sA$ such that $B \in \sigma(\sA_B)$.
\end{nthm}
(A proof is given at the end of this assignment).

Since $\cA = \sigma(\cS)$, we have $\Delta \in \sigma(\cS)$. 
By problem 2.9, there exists some countable sublass $\cS_\Delta$ of $\cS$ such that $\Delta \in \sigma(\cS_\Delta)$. 
%Now it is possible to write out the members of $\cS_\Delta$. 
We conclude
\[
\cS_\Delta = \{ \{\psi_i\} \times \Psi : i \in I_1\} \cup \{ \Psi \times \{ \psi_i \} : i \in I_2 \}
\text{,}
\]
Where $I_1$ and $I_2$ are at most countable indexing sets and could possibly be empty. 
Let $T = \{\psi_i : \psi_i \in \Psi \text{ and }i \in I_1 \cup I_2\}$ so that $T$ is the set of all elements of $\Psi$ that form a singleton cylinder in $\cS_\Delta$.

Let $\cP_1$ be the class containing $T^c$ and the singletons $\{\psi_i\}$ where $i \in I_1 \cup I_2$. 
Hence $\cP_1$ is a countable partition of $\Psi$. 
Let $ \cP_2 = [P_1 \times P_2 : P_1, P_2 \in \cP_1]$.
Hence $\cP_2$ is a countable partition of $\Psi \times \Psi$.
Let $\sU$ (script ``U'') be the class of countable unions of sets (or blocks) in the partition $ \cP_2$, and let $\sU$ include the empty set.
An example, for demonstration, of an element in $\sU$ is the set $(\{\psi_1\} \times \{\psi_1\}) \cup (\{\psi_2\} \times T^c)$. 
We claim (i) $\sU$ is a $\sigma$-field and (ii) $\sigma(\cS_\Delta) \subset \sU$.
\begin{enumerate}[(i)]
\item
By definition, $\sU$ includes the empty set. 
Since $\cP_2$ is countable and since the union of the blocks of $\cP_2$ is $\Psi \times \Psi$, it follows that $\Psi \times \Psi$ is contained in $\sU$. 
A set of $\sU$ is the countable union of blocks of the partition, and that set's complement is the countable union of all the other blocks. 
Thus $\sU$ is closed under complementation.
Finally, $\sU$ is closed under countable union by definition. 
Hence $\sU$ has the three properties of a $\sigma$-field.

\item
The elements of $\cS_\Delta$ are contained in $\sU$ since 
$\{\psi_i\} \in \cP_1$ where $i \in I_1 \cup I_2$, and 
\begin{align*}
\{\psi_i\} \times \Psi 
&= 
\bigcup_{P \in \cP_1}
\left( \{\psi_i\} \times P \right)
=
\{\psi_i\} \times \Psi
\in \sU 
\quad \text{and similarly} 
\\
\Psi \times \{ \psi_i \} 
&= 
\bigcup_{P \in \cP_1}
\left( P \times \{\psi_i\} \right)
=
\Psi \times \{\psi_i\}
\in \sU
\text{.}
\end{align*}
By (i), $\sU$ is a $\sigma$-field containing $\cS_\Delta$. 
Thus $\sigma(\cS_\Delta) \subset \sU$ and $\Delta \in \sU$.
\end{enumerate}
Now we seek the contradiction. Set 
\[
D = \Delta \setminus 
\left(
\bigcup_{\psi \in T} 
(\psi, \psi)
\right) 
=
\bigcup_{\psi \in T^c} (\psi, \psi)
\subset 
T^c \times T^c
\text{.}
\]
Note $\cup_{\psi \in T} (\psi, \psi) = \cup_{\psi \in T} 
(\{\psi\} \times \{\psi\}) \in \sU$ since it is a countable union of elements in $\cP_2$. 
Since $\Delta \in \sU$, we have $D \in \sU$ by the closure properties of $\sU$. 
Since $T^c$ is uncountable, there exist $\psi_1$ and $\psi_2$ in $T^c$ with $\psi_1 \neq \psi_2$. 
It follows that $(\psi_1, \psi_2)$ is an element of $T^c \times T^c$ but it is not an element of $D$. 
Thus $D$ is a strict subset of $T^c \times T^c$.
Since $D \in \sU$, we have $D$ is equal to $U$, some countable union of sets in $\cP_2$. 
One of the sets in the countable union $U$ must be $T^c \times T^c$ (if not, then $D$ is not a subset of $U$). 
In that case, however, $D$ is a strict subset of $T^c \times T^c \subset U$. 
Thus $D \neq U$. This is our contradiction. 
Therefore $\Delta \notin \cA$ as desired.
\end{proof}
%%%%%%%%%%%%%%%%%%%%%%%%%%%%%%%%%%%%%%%%%%%%%%%%%%%%%%%%%%%
%                                                         %
%                     Start Problem 5                     %
%                                                         %
%%%%%%%%%%%%%%%%%%%%%%%%%%%%%%%%%%%%%%%%%%%%%%%%%%%%%%%%%%%
\item
A transition probability is defined as follows:

A mapping $T:\Omega_1\times \cA_2 \to [0,1]$ is called a \textit{transition probability} from $(\Omega_1,\cA_1)$ to $(\Omega_2,\cA_2)$ (briefly, from $\Omega_1$ to $\Omega_2$) if
\begin{enumerate}
\item[(1)] $T(\omega_1;\;\cdot\;)$ is a probability measure on $(\Omega_2,\cA_2)$ for each $\omega_1\in\Omega_1$, and 

\item[(2)] $T(\;\cdot\;;A_2)$ is $\cA_1$-measurable for each $A_2\in\cA_2$.
\end{enumerate} 

Show that given (1), then (2) holds if $T(\;\cdot\; ;A_2)$ is $\cA_1$-measurable for all $A_2$ in some $\pi$-system generating $\cA_2$.

\begin{proof}
Let (1) be true. Let $T(\;\cdot\; ;A_2)$ be $\cA_1$-measurable for all $A_2$ in some $\pi$-system $\sP$ generating $\cA_2$. 
Define 
\[
\sG = \{  A_2 \in \cA_2 : T( \;\cdot \;; A_2) \text{ is $\cA_1$-measurable }\}
\]
We now show that $\sG$ is a $\lambda$-system. 

\textbf{Contains $\Omega_2$:} 
By (1), we have $T(\omega_1; \Omega_2) = 1$ for all $\omega_1 \in \Omega_1$. 
Thus $T( \; \cdot \; ; \Omega_2)$ is a constant function. Hence it is $\cA_1$-measurable, and $\Omega_2 \in \sG$. 

\textbf{Closure under proper difference:} Now suppose $B_1$ and $B_2$ are elements of $\sG$ with $B_1 \subset B_2$. 
Since $B_1$ and $B_2$ are elements of $\cA_2$, their difference $B_2 \setminus B_1$ is an element of $\cA_2$ because $\cA_2$ is a $\sigma$-field.
By (1), it follows $T(\omega_1; B_2 \setminus B_1) = T(\omega_1; B_2) - T(\omega_1; B_1)$ for all $\omega_1 \in \Omega_1$. 
Since, $T( \; \cdot \; ; B_1)$ and $T( \; \cdot \; ; B_2)$ are $\cA_1$-measurable, then, by the corollary to Theorem 3.1.5 in Chung (the closure theorem), their difference, $T(\; \cdot\; ; B_2 \setminus B_1)$, is $\cA_1$-measurable and is therefore a member of $\sG$. 

\textbf{Closure under increasing union:} Finally, let $B_1, B_2, \cdots$ be a sequence of sets in $\sG$ with $B_n \subset B_{n+1}$ for all $n \geq 1$. 
Since $\cA_2$ is a $\sigma$-field, it contains the countable union of the $B_n$.
By (1) and the properties of a probability measure (i.e. monotone sequential continuity from below), for all $\omega_1 \in \Omega_1$ it is true that
\[
\lim_{n \to \infty} T(\omega_1; B_n) = T(\omega_1; \cup_n B_n)
\]
and the limit exists because it is the limit of a bounded monotone increasing.
By Theorem 13.4 in Billingsley, since the limit of $\cA_1$-measurable $T(\; \cdot \; ; B_n)$ exists everywhere (on $\Omega_1$), then that limit, $T(\; \cdot \; ; \cup_n B_n)$, is $\cA_1$-measurable. 
Thus it is in $\sG$. We can conclude that $\sG$ is a $\lambda$-system.

By hypothesis, $\sP \subset \sG$, whence by the $\pi$-$\lambda$ theorem we have $\cA_2 = \sigma(\sP) \subset \sG$. 
By the definition of $\sG$, it is clear $\sG \subset \cA_2$. Therefore $\sG = \cA_2$, which demonstrates (2). 
\end{proof}
%%%%%%%%%%%%%%%%%%%%%%%%%%%%%%%%%%%%%%%%%%%%%%%%%%%%%%%%%%%
%                                                         %
%                     Start Problem 6                     %
%                                                         %
%%%%%%%%%%%%%%%%%%%%%%%%%%%%%%%%%%%%%%%%%%%%%%%%%%%%%%%%%%%
\item
Let $(\Omega_1,\cA_1),(\Omega_2,\cA_2)$ be measurable spaces, $\Omega=\Omega_1\times\Omega_2$, $\cA=\cA_1\otimes\cA_2$. 
Let $M$ be a probability on $(\Omega_1,\cA_1)$, and let $(T_{\omega_1})_{\omega_1\in\Omega_1}$ be a transition probability from $\Omega_1$ to $\Omega_2$. 
Let $\sG = \{ A\in\cA:\omega_1\mapsto T_{\omega_1}(A_{\omega_1})$ is $\cA_1$-measurable$\}$. 
Show that
\begin{equation} 
\label{eq3}
T_{\omega_1}((A_1\times A_2)_{\omega_1}) = I_{A_1}(\omega_1)T_{\omega_1}(A_2).
\end{equation}
Also show $\sG$ contains the $\pi$-system $\cR$.
\begin{proof}
Calculating, 
\[
T_{\omega_1}((A_1 \times A_2)_{\omega_1})
= T(\omega_1; (A_1 \times A_2)_{\omega_1})
= 
\begin{cases}
T(\omega_1; \emptyset) = 0 & \text{if $\omega_1 \notin A_1$} \\
T(\omega_1; A_2) = T_{\omega_1}(A_2) & \text{if $\omega_1 \in A_1$}
\end{cases}
\text{.}
\]
Since,
\[
I_{A_1}(\omega_1) T_{\omega_1}(A_2)
=
\begin{cases}
0 & \text{if $\omega_1 \notin A_1$} \\
T_{\omega_1}(A_2) & \text{if $\omega_1 \in A_1$}
\end{cases}
\text{,}
\]
we have \eqref{eq3}, as desired.

To show $\sG$ contains $\cR$, let $R = A_1 \times A_2 \in \cR$ with $A_1 \in \cA_1$ and $A_2 \in \cA_2$. Our task is to show that the function on $\Omega_1$ that sends $\omega_1\mapsto T_{\omega_1}(R_{\omega_1})$ is an $\cA_1$-measurable function. 
By \eqref{eq3}, we have
\[
T_{\omega_1}(R_{\omega_1}) = T_{\omega_1}((A_1\times A_2)_{\omega_1}) = I_{A_1}(\omega_1)T_{\omega_1}(A_2).
\]
Since $T_{\omega_1}$ is a transition probability, condition (2) holds that $T(\;\cdot\;;A_2)$ is $\cA_1$-measurable for each $A_2\in\cA_2$. In particular, for the $A_2$ in $R = A_1 \times A_2$, we have $T_{\omega_1}(A_2)=T(\;\cdot\;;A_2)$ is $\cA_1$-measurable.
%Since $T_{\omega_1}$ is a transition probability, for all $\omega_1 \in \Omega_1$, we have $T_{\omega_1}(A_2)$ is constant. 
Also since $A_1 \in \cA_1$, the indicator $I_{A_1}$ is $\cA_1$-measurable. Note the product of $\cA_1$-measurable functions is measurable by the corollary to Theorem 3.1.5 in Chung (the closure theorem). 
Hence $T_{\omega_1}(R_{\omega_1})$ is $\cA_1$-measurable and $R \in \sG$. Therefore, $\cR \subset \sG$. 
\end{proof}

%%%%%%%%%%%%%%%%%%%%%%%%%%%%%%%%%%%%%%%%%%%%%%%%%%%%%%%%%%%
%                                                         %
%                     Start Problem 7                     %
%                                                         %
%%%%%%%%%%%%%%%%%%%%%%%%%%%%%%%%%%%%%%%%%%%%%%%%%%%%%%%%%%%
\item
Notations are the same as Problem 6. Show that $\sG$ is a $\lambda$-system.
\begin{proof}
\textbf{Contains $\Omega$:} Since for all $\omega_1 \in \Omega_1$, we have $T_{\omega_1}(\Omega_{\omega_1}) = T_{\omega_1}(\Omega_2) = 1$, then $\omega_1\mapsto T_{\omega_1}(\Omega_{\omega_1})$ is a constant function. Thus it is $\cA_1$-measurable, and $\Omega \in \sG$.

\textbf{Closure under complementation:} Let $A \in \sG$. Thus $\omega_1\mapsto T_{\omega_1}(A_{\omega_1})$ is $\cA_1$-measurable. Then, since sectioning commutes with set operations, 
\[
\omega_1 \mapsto T_{\omega_1} ((A^c)_{\omega_1}) = T_{\omega_1}((A_{\omega_1})^c) = 1 - T_{\omega_1} (A_{\omega_1})
\]
is $A_1$-measurable by the closure theorem.

\textbf{Closure under countable union of disjoint sets: } Let $B_1, B_2, \cdots$ be mutually disjoint elements of $\sG$. Then for all $\omega_1 \in \Omega_1$ and for all integers $I \geq 1$ we have
\begin{align*}
T_{\omega_1} 
\left(
\left( 
\bigcup_{i = 1}^I B_i
\right)_{\omega_1} 
\right) 
&= T_{\omega_1}
\left(
\bigcup_{i=1}^I (B_i)_{\omega_1} 
\right) 
\quad \text{since sectioning commutes}
\\
&=
\sum_{i=1}^I
T_{\omega_1}
\left(
(B_i)_{\omega_1} 
\right) 
\end{align*}
since $T(\omega_1;\;\cdot\;) = T_{\omega_1}(\;\cdot\;)$ is a probability measure on $(\Omega_2,\cA_2)$ and the $B_i$ are disjoint. For all $\omega_1$,
\begin{align}
\lim_{I \to \infty}
T_{\omega_1} 
\left(
\left( 
\bigcup_{i = 1}^I B_i
\right)_{\omega_1} 
\right) 
&=
\lim_{I \to \infty}
\sum_{i=1}^I
T_{\omega_1}
\left(
(B_i)_{\omega_1} 
\right) 
\label{pt1}
\end{align}
The series
converges since it is the limit of a bounded (by 1), monotone increasing sequence of partial sums.

Calculating, for all $\omega_1$
\begin{align}
T_{\omega_1}
\left(
\left(
\lim_{I \to \infty}
\bigcup_{i=1}^I B_i
\right)
_{\omega_1}
\right)
&=
T_{\omega_1}
\left(
\lim_{I \to \infty}
\left(
\bigcup_{i=1}^I B_i
\right)
_{\omega_1}
\right)
\quad\text{since sectioning commutes}
\notag
\\
&=
\lim_{I \to \infty}
T_{\omega_1} 
\left(
\left( 
\bigcup_{i = 1}^I B_i
\right)_{\omega_1} 
\right) 
\label{pt2}
\end{align}
by monotone sequential continuity from below.

Therefore by \eqref{pt1} and \eqref{pt2} and Theorem 13.4 (ii) in Billingsley,
we have  
\[
\omega_1 \mapsto T_{\omega_1} \left(\left( \lim_{I \to \infty}\bigcup_{i = 1}^I B_i\right)_{\omega_1} \right)
\]
is measurable. Hence, $\lim_{I \to \infty}\cup_{i = 1}^I B_i \in \sG$.
We conclude $\sG$ is a $\lambda$-system.
\end{proof}

%%%%%%%%%%%%%%%%%%%%%%%%%%%%%%%%%%%%%%%%%%%%%%%%%%%%%%%%%%%
%                                                         %
%                     Start Problem 8                     %
%                                                         %
%%%%%%%%%%%%%%%%%%%%%%%%%%%%%%%%%%%%%%%%%%%%%%%%%%%%%%%%%%%
\item
Notations are the same as Problem 6. 
Let $M$ be a probability on $(\Omega_1, \cA_1)$. 
The set function $MT$, defined on $\cA$ is 
\[
MT(A) := \int_{\Omega_1} T_{\omega_1} (A_{\omega_1}) M (d \omega_1)
\text{.}
\]
Furthermore, for any nonnegative $\cA$-random variable $X$ on $\Omega$, the map $TX$ on $\Omega_1$ is defined as 
\[
\omega_1 \mapsto \int_{\Omega_2} X_{\omega_1} (\omega_2) T_{\omega_1} (d \omega_2)
\]
Let $\cG$ denote the collection of $\cA$-measurable nonnegative random variables $X$ such that $\omega_1\mapsto\int_{\Omega_2}X_{\omega_1}(\omega_2)T_{\omega_1}(\text{d}\omega_2)$ is $A_1$-measurable and $\langle MT,X\rangle= \langle M,TX \rangle$. 
Show that $I_A\in\cG$ for every $A\in\cA$.

\begin{proof}
Let $A \in \cA$.
By problem 2 of this homework, we have $(I_A)_{\omega_1} = I_{A_{\omega_1}}$. Thus
\begin{align}
\omega_1 \mapsto
\int_{\Omega_2} (I_A)_{\omega_1}(\omega_2) T_{\omega_1}(d\omega_2)
&= \int_{\Omega_2} I_{A_{\omega_1}}(\omega_2) T_{\omega_1}(d\omega_2) \label{eq4} \\
&= \int_{A_{\omega_1}} T_{\omega_1} (d \omega_2) \\
&= T_{\omega_1} (A_{\omega_1}) 
\quad
\text{since $T_{\omega_1}$ is a probability measure}\label{eq6}
\end{align}
is $\cA_1$-measurable by the reasoning that follows after problem 7 in the course slides (that $\sG = \cA$).

For the second part, calculating gives
\begin{align*}
\langle MT, I_A \rangle 
&= \int_{\Omega} I_A(\omega) MT(d\omega) \\
&= \int_A MT(d \omega) \\
&= MT(A) 
\quad
\text{since $MT$ is a probability measure}\\
&= \int_{\Omega_1} T_{\omega_1} (A_{\omega_1}) M (d \omega_1)
\quad \text{ by the definition of $MT$} \\
&= \int_{\Omega_1} \left( \int_{\Omega_2} I_{A_{\omega_1}}(\omega_2) T_{\omega_1}(d\omega_2) \right) M (d \omega_1)
\quad \text{by $\eqref{eq4} = \eqref{eq6}$}\\
&= \int_{\Omega_1} (TI_A)(\omega_1) M(d\omega_1) 
\quad \text{by the definition of $TI_A$, see bottom pg. 51}\\
&= \langle M, TI_A \rangle
\end{align*}
Thus $I_A \in \cG$.
\end{proof}
\end{enumerate}

\section*{Acknowledgements}
I helped Claire, Leslie, David, and Vivek with Billingsley 2.9. I worked with David and Vivek on the assignment.

\newpage
\section*{Billingsley 2.9}
Show that, if $B \in \sigma(\sA)$, then there exists a countable subclass $\sA_B$ of $\sA$ such that $B \in \sigma(\sA_B)$.
\begin{proof}
If $\sA$ is at most countable, then take $\sA_B = \sA$. So suppose $\sA$ is uncountable. Define
\[
\cZ = \{ \zeta : \zeta \subset \sA
\text{ and $\zeta$ is at most countable}
\}
\]
Here ``at most countable'' means empty, finite, or countable. Note $\cZ$ is nonempty.
Let
\[
\cF = \bigcup_{\zeta \in \cZ} \sigma(\zeta)
\]
Then $\cF$ is union of all $\sigma$-fields generated by an element of $\cZ$. The goal is to show that $\cF = \sigma(\sA)$.

Given $\zeta \in \cZ$, it follows that $\Omega \in \sigma(\zeta)$ by properties of $\sigma$-field. Therefore $\Omega \in \cF$ because $\sigma(\zeta) \subset \cF$.

Let $A \in \cF$. Then exists a $\zeta_0 \in \cZ$ such that $A \in \sigma(\zeta_0)$. By properties of $\sigma$-field, $A^c \in \sigma(\zeta_0)$. Thus
\[
A^c \in \sigma(\zeta_0) \subset \cF
\]

Suppose $A_1, A_2, \cdots$ is a sequence of elements in $\cF$. Then there exist $\zeta_1, \zeta_2, \cdots$ in $\cZ$ such that $A_i \in \sigma(\zeta_i)$ for all $i$. Since the countable union of countable sets is itself countable, it follows that
$
\cup_{j=1}^\infty \zeta_j \in \cZ
$.
For all $i$,
\[
\zeta_i 
\subset 
\bigcup_{j=1}^\infty \zeta_j
\subset
\sigma
\left(
\bigcup_{j=1}^\infty \zeta_j
\right)
\subset
\cF
\]
Therefore, since $\sigma(\zeta_i)$ is a subset of all $\sigma$-fields that contain $\zeta_i$, it follows that 
\[
\sigma(\zeta_i) 
\subset 
\sigma
\left(
\bigcup_{j=1}^\infty \zeta_j
\right)
\]
for all $i$. Hence, 
\[
A_i 
\in 
\sigma(\zeta_i)
\subset
\sigma
\left(
\bigcup_{j=1}^\infty \zeta_j
\right)
\]
for all $i$. By properties of $\sigma$-field, 
\[
\bigcup_{i=1}^\infty 
A_i
\in
\sigma
\left(
\bigcup_{j=1}^\infty \zeta_j
\right)
\subset
\cF
\]
Thus $\cF$ is closed under countable union, and it has been demonstrated that $\cF$ is a $\sigma$-field.

Suppose $A \in \sA$. Then $\{A\} \in \cZ$ and it follows that $A \in \sigma(\{A\}) \subset \cF$.  Thus $\cF$ is a $\sigma$-field that contains $\sA$. Hence $\sigma(\sA) \subset \cF$. Suppose $F \in \cF$. Then there exists $\zeta \in \cZ$ such that $F \in \sigma(\zeta)$. Since $\zeta \subset \sA \subset \sigma(\sA)$, it follows that $\sigma(\zeta) \subset \sigma(\sA)$. 
Therefore, $F \in \sigma(\zeta) \subset \sigma(\sA)$ and $\cF \subset \sigma(\sA)$. Since the set inclusion has been shown in both directions,
\[
\cF = \sigma(\sA)
\]
As shown in the paragraph above, given any $B \in \cF = \sigma(\sA)$, there exists $\zeta \in \cZ$, some subset of $\sA$, such that $B \in \sigma(\zeta)$. Define $\sA_B := \zeta$, and it is clear that $\sA_B$ is a countable subclass of $\sA$.
\end{proof}

\end{document}