%%%%%%%%%%%%%%%%%%%%%%%%%%%%%%%%%%%%%%%%%%%%%
%%%%%%%%%%%%%%%%%%%%%%%%%%%%%%%%%%%%%%%%%%%%%
%%%THIS IS STANDARD PREAMBLE-TYPE STUFF%%%
\documentstyle[leqno]{article}

\addtolength{\textheight}{+1in}
\setlength{\textwidth}{5.75in}
\setlength{\oddsidemargin}{0.5in}
\addtolength{\topmargin}{-\topmargin}

\nofiles

\pagestyle{myheadings}
\markright{{\rm  550.621 Final Exam, Spring 2013}}

\title{550.621 Final Examination, Spring 2013}
\author{James Allen Fill}
\date{}
%%%  THE NEXT LINE IS FOR MAKING TRANSPARENCIES  %%%
%%%  \renewcommand{\baselinestretch}{2}

\begin{document}
\newtheorem{theorem}{Theorem}[section]
\newtheorem{corollary}[theorem]{Corollary}
\newtheorem{lemma}[theorem]{Lemma}
\newtheorem{proposition}[theorem]{Proposition}
\newtheorem{example}[theorem]{\sl Example}
\newtheorem{remark}[theorem]{\sl Remark}
%%%  NEW ONES            %%%
\newtheorem{definition}[theorem]{\sl Definition}
%%%%%%%%%%%%%%%%%%%%%%%%%%%%
\newcommand{\qed}{\ \ \rule{1ex}{1ex}}
\newcommand{\lc}{\left\lceil}
\newcommand{\lf}{\left\lfloor}
\newcommand{\rc}{\right\rceil}
\newcommand{\rf}{\right\rfloor}
\newcommand{\proof}{\par\noindent  {\bf  Proof.  }}
\newcommand{\bd}[1]{{\bf  #1}}
\newcommand{\RR}{{\bf  R}}
\newcommand{\NN}{{\bf  N}}
\newcommand{\ZZ}{{\bf  Z}}
\newcommand{\CC}{{\bf  C}}
\newcommand{\QQ}{{\bf  Q}}
\newcommand{\SS}{{\bf  S}}
\newcommand{\xx}{{\bf  x}}
\newcommand{\yy}{{\bf  y}}
\newcommand{\zz}{{\bf  z}}
\newcommand{\ww}{{\bf  w}}
\newcommand{\Sj}{{{\cal S}_j}}
\newcommand{\Pt}{{\tilde{P}}}
\newcommand{\Var}{{\rm Var}}
\newcommand{\diag}{{\rm diag}}
\newcommand{\Exp}{{\rm Exp}}
\newcommand{\ave}{{\rm E}}
\newcommand{\Xt}{{\tilde{X}}}
\newcommand{\SSS}{{S}} %I'd like to replace this by a German S for the symm. 
%                        gp.
\newcommand{\gxy}{g_{xy}}
\newcommand{\Gamxy}{{\Gamma(x, y)}}
\newcommand{\vece}{{\vec{e}}}
\newcommand{\vecep}{{\vece\,'}}
\newcommand{\Qe}{{Q(\vece\,)}}
\newcommand{\Qep}{{Q(\vecep)}}
\newcommand{\we}{{w(\vece\,)}}
\newcommand{\wep}{{w(\vecep)}}
\newcommand{\Kwe}{{K_w(\vece\,)}}
\newcommand{\phie}{{\phi(\vece\,)}}
\newcommand{\psie}{{\psi(\vece\,)}}
\newcommand{\be}{{b(\vece\,)}}
\newcommand{\vecE}{{\vec{E}}}
\newcommand{\GamxyQ}{{|\Gamma(x, y)|_Q}}
\newcommand{\Gamxyw}{{|\Gamma(x, y)|_w}}
\newcommand{\gamstar}{{\gamma_*}}
\newcommand{\gamstart}{{\gamma^t_*}}
\newcommand{\gamstarone}{{\gamma^1_*}}
\newcommand{\gamstartwo}{{\gamma^2_*}}
\newcommand{\gamstarthree}{{\gamma^3_*}}
\newcommand{\gamstarfour}{{\gamma^4_*}}
\newcommand{\bte}{{b^t(\vece\,)}}
\newcommand{\bonee}{{b^1(\vece\,)}}
\newcommand{\btwoe}{{b^2(\vece\,)}}
\newcommand{\bthreee}{{b^3(\vece\,)}}
\newcommand{\bfoure}{{b^4(\vece\,)}}
\newcommand{\bthreeae}{{b^{3\mbox{\scriptsize a}}(\vece\,)}}
\newcommand{\bthreebe}{{b^{3\mbox{\scriptsize b}}(\vece\,)}}
\newcommand{\bfourae}{{b^{4\mbox{\scriptsize a}}(\vece\,)}}
\newcommand{\bfourbe}{{b^{4\mbox{\scriptsize b}}(\vece\,)}}
\newcommand{\Ac}{{\cal A}}
\newcommand{\Bc}{{\cal B}}
\newcommand{\Cc}{{\cal C}}
\newcommand{\Ec}{{\cal E}}
\newcommand{\Fc}{{\cal F}}
\newcommand{\Gc}{{\cal G}}
\newcommand{\Lc}{{\cal L}}
\newcommand{\Nc}{{\cal N}}
\newcommand{\Pc}{{\cal P}}
\newcommand{\Sc}{{\cal S}}
\newcommand{\Ft}{\Fc_t}
\newcommand{\Fi}{\Fc_{\infty}}
\newcommand{\Fn}{\Fc_n}
\newcommand{\Tt}{{\tilde{T}}}
\newcommand{\ft}{{\tilde{f}}}
\newcommand{\Jt}{{\tilde{J}}}
\newcommand{\tp}{t+}
\newcommand{\tm}{t-}

%%%  NEW ONES START HERE  %%%
\newcommand{\vdn}{\| \pi_n - \pi \|}
\newcommand{\vdt}{\| \pi_t - \pi \|}
\newcommand{\sep}{{\rm sep}}
\newcommand{\Fets}{\Fc^*_{=t}}
\newcommand{\ppi}{{\mbox{\boldmath $\pi$}}}
\newcommand{\PP}{{\bf  P}}
\newcommand{\PPh}{\PP^{(h)}}
\newcommand{\GG}{{\bf  G}}
\newcommand{\II}{{\bf  I}}
\newcommand{\XX}{{\bf  X}}
\newcommand{\Qsp}{{Q^*}'}
\newcommand{\sn}{\sigma \sqrt{n}}
\newcommand{\tn}{\tau_n}
\newcommand{\tnm}{\tau_{n-1}}
\newcommand{\gt}{{\tilde{g}}}
\newcommand{\Gt}{{\tilde{G}}}
\newcommand{\pimin}{\pi_{\min}}
\newcommand{\supf}{\left( \sup_{\delta \leq |t| < \pi} |f(t)| \right)^{n-1}}
%%%%%%%%%%%%%%%%%%%%%%%%%%%%%

\newcommand{\bege}{\begin{equation}}
\newcommand{\ene}{\end{equation}}
\newcommand{\begp}{\begin{proposition}}
\newcommand{\enp}{\end{proposition}}
\newcommand{\begt}{\begin{theorem}}
\newcommand{\ent}{\end{theorem}}
\newcommand{\begl}{\begin{lemma}}
\newcommand{\enl}{\end{lemma}}
\newcommand{\begc}{\begin{corollary}}
\newcommand{\enc}{\end{corollary}}
\newcommand{\begr}{\begin{remark}}
\newcommand{\enr}{\end{remark}}
%%%  NEW ONES             %%%
\newcommand{\begd}{\begin{definition}}
\newcommand{\enf}{\end{definition}}
\newcommand{\begx}{\begin{example}}
\newcommand{\enx}{\end{example}}
%%%%%%%%%%%%%%%%%%%%%%%%%%%%%
\newcommand{\bega}{\begin{array}}
\newcommand{\ena}{\end{array}}


%%% SOME OF MHS's STUFF FOLLOWS   %%%
\newcommand{\Line}{$\underline{\qquad\qquad\qquad\qquad}$}
\newcommand{\X}{$\qquad$}
%	brackets, parenthesis, and other miscelaneous marks
\newcommand{\lb}{\left\{}
\newcommand{\rb}{\right\}}
\newcommand{\lsb}{\left[}
\newcommand{\rsb}{\right]}
\newcommand{\lp}{\left(}
\newcommand{\rp}{\right)}
\newcommand{\ls}{\left|}
\newcommand{\rs}{\right|}
\newcommand{\lss}{\left\|}
\newcommand{\rss}{\right\|}

%	struts and spaces
\newcommand{\vs}{\vspace{\smallskipamount}\noindent}
\newcommand{\vm}{\vspace{\medskipamount}\noindent}
\newcommand{\vb}{\vspace{\bigskipamount}\noindent}

%	relations and implications
\newcommand{\ra}{\rightarrow}
\newcommand{\la}{\leftarrow}
\newcommand{\implies}{\Longrightarrow}
% END OF MHS's STUFF


%%%THIS IS THE END OF STANDARD PREAMBLE-TYPE STUFF%%%
%%%%%%%%%%%%%%%%%%%%%%%%%%%%%%%%%%%%%%%%%%%%%%%%%%%%%%%
\section*{550.621 Final Examination, Spring 2013}

\subsection*{Official Examination Policy}

\hspace{\parindent}
This take-home final examination for 550.621 Probability Theory~II is due {\em
no later than\/} {\bf 5:00 PM~EDT, Thursday, May~16, 2013}.  There
are only three acceptable ways of turning in your exam: (1) Hand it to me in
person, or (2) slide it under the door of my office, 306-F Whitehead Hall, or
(3) submit it electronically to me at {\tt jimfill@jhu.edu}.  If you choose option~(3),
please \emph{also} submit an electronic copy to Teaching Assistant Jason Matterer at 
{\tt jmatter4@jhu.edu}. 

You are not to discuss the exam with anyone except me before 
the due deadline.  Conversation, even on the most casual level, about the exam
will be considered a violation of the university's honor code.

Your work should be legibly written in complete English sentences (and paragraphs!). 
Every solution must be clearly explained.  Be sure that your name is on every page of
your solutions.  Please write on only one side of each page.

You may consult your class notes and the course texts by
Billingsley and Chung.  You may also consult references on elementary
probability and real analysis; these should not, however, make use of
measure theory.  {\em You may not use any other references of any kind.\/}

Point values for each question are shown in parentheses.

{\bf Your exam will be graded on a 300-point basis.}

\subsection*{Examination Problems (300 points total)}

\begin{enumerate}

\item {\bf (100 total)} Let $X_j$, $j \geq 1$, be a sequence of independent
random variables, and set $S_n := \sum_{j=1}^n X_j$.  Suppose $\alpha \geq 1$
and 
$$
X_j = \left\{ \begin{array}{ll}
               \pm j^{\alpha}, & \mbox{with probability\ }\frac{1}{4
                                  j^{2(\alpha - 1)}} \mbox{\ each;} \\
               0,              & \mbox{with probability\ }1 - \frac{1}{2
                                  j^{2(\alpha - 1)}}.
              \end{array}
      \right.
$$  
 \begin{enumerate}
\item {\bf (25)} For precisely which values of $\alpha$ does the sequence
$(S_n)$ converge almost surely to an almost surely finite limit? 
\item {\bf (25)} Calculate $s_n^2 := \sum_{j=1}^n \Var(X_j)$ for each $n$.
Describe the associated normalized double
array, and prove by considering variances that it is holospoudic for
any value of $\alpha$.
\item {\bf (25)} For precisely which values of $\alpha$ is Lindeberg's condition
satisfied?
\item {\bf (25)} For precisely which values of $\alpha$ does $S_n/s_n$ converge
in distribution to standard normal?  [Here's something to think about after the
exam.  Does $S_n/s_n$ have {\em any\/} limiting distribution if not standard
normal?  If so, what?]
 \end{enumerate}

\item {\bf (70)} Let $(\Omega,\Fc,P)$ be a probability space, let $\Fc_0$,
$\Fc_1$, and $\Fc_2$ be three sub-$\sigma$-fields of $\Fc$, and suppose that
$\Fc_0 \subset \Fc_1$.  Show that
\bege
\mbox{$\forall A \in \Fc_1$\ \ $\forall B \in \Fc_2$:\ \ $P(A \cap B|\Fc_0) =
P(A|\Fc_0)\ P(B|\Fc_0)$\,\ a.s.}
\ene
if and only if
\bege
\mbox{$\forall B \in \Fc_2$:\ \ $P(B|\Fc_0) = P(B|\Fc_1)$\,\ a.s.} 
\ene

\item {\bf (130 total)} Let $S_n = \sum_{j=1}^n X_j$, where the $X_j$'s are
independent random variables with a common distribution function $F$ of the
integer lattice type with span $1$.  Suppose that $X_1$ has mean $0$ and
variance $\sigma^2 \in (0,\infty)$.  The main objective of this problem is to
prove the {\em local central limit theorem\/} result that
\begin{equation}
\sn \left[ P\left\{ \frac{S_n}{\sn} = \frac{j}{\sn} \right\} - \frac{1}{\sn}
\varphi\!\left( \frac{j}{\sn} \right) \right] \longrightarrow 0 \mbox{\ \
uniformly in $j \in \ZZ$} 
\end{equation}
as $n \to \infty$, where $\varphi$ denotes the standard normal density
function.  
 \begin{enumerate}
\item {\bf (25)} Grant~(3) for the moment and use it to derive the {\em global
central limit theorem\/} for $S_n$: for real numbers $-\infty<a<b<\infty$,
$$
P\left\{ a < \frac{S_n}{\sn} \leq b \right\} \longrightarrow P\{a < Z \leq b\}
$$
as $n \to \infty$, where $Z$ has a standard normal distribution.
\item {\bf (40)} Let $f$ denote the characteristic function of $X_1$.  Begin the
proof of (3) by showing that the expression there equals $1/(2 \pi)$ times
\begin{equation}
\int_{|u|< \pi \sn} \left[ \left( f\left(\frac{u}{\sn} \right) \right)^n -
e^{-\frac{1}{2}u^2} \right] e^{-ji \frac{u}{\sn}}\,du - \int_{|u| \geq \pi
\sn} e^{-\frac{1}{2} u^2} e^{-ji \frac{u}{\sn}}\,du.
\end{equation}
\item {\bf (10)} Show that the second integral in (4) tends to $0$ uniformly in $j
\in \ZZ$ as $n \to \infty$.
\item {\bf (55)} The first integral in (4) is dominated by the integral
$$
\int_{|u|< \pi \sn} \left| \left( f\left(\frac{u}{\sn} \right) \right)^n -
e^{-\frac{1}{2}u^2} \right|\,du.
$$
Complete the proof of (3) by showing that this integral tends to $0$ as $n
\to \infty$.  [{\sc hint:}  Divide the range $|u| < \pi \sn$ into the three
regions $|u| \leq M$, $M < |u| < \delta \sn$, and $\delta \sn \leq |u| < \pi
\sn$ for suitably chosen constants $M$ and $\delta$.  Show that each of the
three corresponding integrals is small.]   
 \end{enumerate}
 \end{enumerate}
 \bigskip
 \bigskip
 
 I would make the following problem a 100-point exam problem except that its solution
 may be found both in Billingsley's book and in Chung's.  I regret that I don't have
 a martingale problem to replace it on the exam!

\begin{enumerate}

\item[4.] (Convergence Theorem for Backward Martingales.)  Let $(X_n)_{n
\leq 0}$ be a martingale with respect to $(\Fc_n)_{n \leq 0}$.  Then (with no
conditions!)\ there exists an integrable random variable $X_{- \infty}$ such
that $X_n$ converges, both in L$^1$ and a.s.,\ to $X_{- \infty}$ as $n \to -
\infty$.  Prove this.

\end{enumerate}

\end{document}






















