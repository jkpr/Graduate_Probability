% /**
%  * A template for homework files in math classes. The 
%  * packages and newcommands are a good starting point.
%  *
%  * Author: James K. Pringle
%  * E-mail: jameskpringle@gmail.com
%  * Last Changed: 5 September 2013
%  *
%  * "LaTeX countains the increasing union of MS Word"
%  */
%~~~~~~~~~~~~~~~~~~~~~~~~~~~~~~~~~~~~~~~~~~~~~~~~~~~~~~~~~%
%%%%%%%%%%%%%%%%%%%%%%%%%%%%%%%%%%%%%%%%%%%%%%%%%%%%%%%%%%%
%                                                         %
%                        PAGE SETUP                       %
%                                                         %
%%%%%%%%%%%%%%%%%%%%%%%%%%%%%%%%%%%%%%%%%%%%%%%%%%%%%%%%%%%
\documentclass[letterpaper, 12pt]{article}

% 1in margins all the way around
\usepackage[margin=1in]{geometry}

% Sets \parindent to 0 and \parskip to stretchable.
\usepackage{parskip}
% Use for bigger spaces between paragraphs.
%\parskip=1.5\baselineskip

% Set headers and footers
\usepackage{fancyhdr}
\pagestyle{fancy}
% Header
\renewcommand{\headrulewidth}{0.4pt}
\lhead{\textsc{\mathclass}}
\chead{\textsc{\today}}
\rhead{\textsc{\mynamehdr}}
% Footer
\renewcommand{\footrulewidth}{0.4pt}
\lfoot{}
\cfoot{\thepage}
\rfoot{}

% Make the space between lines slightly more generous 
% than normal single spacing, but compensate so that the 
% spacing between rows of matrices still looks normal.  
% Note that 1.1=1/.9090909...
\renewcommand{\baselinestretch}{1.1}
\renewcommand{\arraystretch}{.91}

%%%%%%%%%%%%%%%%%%%%%%%%%%%%%%%%%%%%%%%%%%%%%%%%%%%%%%%%%%%
%                                                         %
%                      USEFUL PACKAGES                    %
%                                                         %
%%%%%%%%%%%%%%%%%%%%%%%%%%%%%%%%%%%%%%%%%%%%%%%%%%%%%%%%%%%

% The classic three
\usepackage{amsmath,amsthm,amssymb}

% Define \newtheorem for use
% No numbers, labeled 'Theorem'
\newtheorem*{nthm}{Theorem}

% Not sure what this is for
\usepackage{amsfonts}

% Fancy script font
\usepackage{mathrsfs}

% Makes enumerate environment much easier to customize
% by specifying the counter
\usepackage{enumerate}

% Color
\usepackage{color}
\usepackage[usenames,dvipsnames,svgnames,table]{xcolor}

% URL links
\usepackage{hyperref}

% For inserting graphics and images
\usepackage{graphicx}
\usepackage{float}
\usepackage[footnotesize]{caption}



%%%%%%%%%%%%%%%%%%%%%%%%%%%%%%%%%%%%%%%%%%%%%%%%%%%%%%%%%%%
%                                                         %
%                   USER-DEFINED COMMANDS                 %
%                                                         %
%%%%%%%%%%%%%%%%%%%%%%%%%%%%%%%%%%%%%%%%%%%%%%%%%%%%%%%%%%%

% Make a hyperlink with colored text
\newcommand{\hrefcolor}[3]{\href{#1}{\textcolor{#3}{#2}}}

% Make a hyperlink with gray text
\newcommand{\hrefgray}[2]{\hrefcolor{#1}{#2}{Gray}}

% Make the header for the first page
\newcommand{\firstpageinfo}{
\textsf{
\begin{flushleft}
\sc \myname \\
\normalfont \mathclass \\
\professorname \\
\assignmentnumber \\
\thedate
\end{flushleft}
} \bigskip
}

% Make problem list for "title" of page
\newcommand{\problemlist}{ 
\begin{center}
\textbf{\Large \textsf{\assignmentnumber}}\\
\textit{\textsf{\problemset}}
\end{center}
\bigskip
}

%~~~~~~~~~~~~~~~~~~~~~~~~~~~~~~~~~~~~~~~~~~~~~~~~~~~~~~~~~%
%                                                         %
%               LETTERS, FUNCTIONS, AND TEXT              %
%                                                         %
%~~~~~~~~~~~~~~~~~~~~~~~~~~~~~~~~~~~~~~~~~~~~~~~~~~~~~~~~~%

% A
\newcommand{\cA}{\mathcal{A}}
\newcommand{\sA}{\mathscr{A}}
\renewcommand{\aa}{\;\text{a.a.}}
\renewcommand{\ae}{\;\text{a.e.}}
% B
\newcommand{\B}{\mathscr{B}}
\newcommand{\cB}{\mathcal{B}}
% C
\newcommand{\cC}{\mathcal{C}}
\newcommand{\cov}{\text{cov}}
% E
\newcommand{\E}{\mathbb{E}}
% F
\newcommand{\sF}{\mathscr{F}}
\newcommand{\cF}{\mathcal{F}}
\newcommand{\Ft}{F^\sim}
% G
\newcommand{\cG}{\mathcal{G}}
\newcommand{\sG}{\mathscr{G}}
% I
\newcommand{\io}{\;\text{i.o.}}
% N
\newcommand{\N}{\mathbb{N}}
% P
\newcommand{\cP}{\mathcal{P}}
\newcommand{\sP}{\mathscr{P}}
\newcommand{\pr}{\text{pr}}
% Q
\newcommand{\Q}{\mathbb{Q}}
% R
\newcommand{\R}{\mathbb{R}}
\newcommand{\bR}{\mathbf{R}}
\newcommand{\cR}{\mathcal{R}}
% S
\newcommand{\cS}{\mathcal{S}}
% U
\newcommand{\cU}{\mathcal{U}}
% V
\newcommand{\var}{\text{var}}
% Z
\newcommand{\cZ}{\mathcal{Z}}
\newcommand{\Z}{\mathbb{Z}}
% Punctuation
\newcommand{\sbs}{\;|\;} % space bar space
% Math
\newcommand{\imii}{\int_{-\infty}^\infty}
\newcommand{\pion}{\prod_{i=1}^n}
\newcommand{\pioI}{\prod_{i=1}^I}
\newcommand{\pjon}{\prod_{j=1}^n}
\newcommand{\pjoJ}{\prod_{j=1}^J}
\newcommand{\pkon}{\prod_{k=1}^n}
\newcommand{\pkoK}{\prod_{k=1}^K}
\newcommand{\sion}{\sum_{i=1}^n}
\newcommand{\sioI}{\sum_{i=1}^I}
\newcommand{\sjon}{\sum_{j=1}^n}
\newcommand{\sjoJ}{\sum_{j=1}^J}
\newcommand{\skon}{\sum_{k=1}^n}
\newcommand{\skoK}{\sum_{k=1}^K}
\newcommand{\sioi}{\sum_{i=1}^\infty}
\newcommand{\sjoi}{\sum_{j=1}^\infty}
\newcommand{\skoi}{\sum_{k=1}^\infty}
\newcommand{\sio}{\sum_{i=1}}
\newcommand{\sjo}{\sum_{j=1}}
\newcommand{\sko}{\sum_{k=1}}
% Typography
\newcommand{\scb}[1]{\textsc{\textbf{#1}}}

%~~~~~~~~~~~~~~~~~~~~~~~~~~~~~~~~~~~~~~~~~~~~~~~~~~~~~~~~~%
%                                                         %
%            CHANGE THESE BASED ON THE PAPER              %
%                                                         %
%~~~~~~~~~~~~~~~~~~~~~~~~~~~~~~~~~~~~~~~~~~~~~~~~~~~~~~~~~%

% Constants for fancy header and first page info
\newcommand{\mynamehdr}{\hrefgray{http://biostat.jhsph.edu/~jpringle/}{\myname}}
\newcommand{\mathclass}{550.621 Probability}
\newcommand{\myname}{James K. Pringle}
\newcommand{\professorname}{Dr. Jim Fill}
\newcommand{\assignmentnumber}{Assignment 2}
\newcommand{\thedate}{\today}
\newcommand{\problemset}{Billingsley 2.9}

%%%%%%%%%%%%%%%%%%%%%%%%%%%%%%%%%%%%%%%%%%%%%%%%%%%%%%%%%%%
%                                                         %
%                      BEGIN DOCUMENT                     %
%                                                         %
%%%%%%%%%%%%%%%%%%%%%%%%%%%%%%%%%%%%%%%%%%%%%%%%%%%%%%%%%%%
\begin{document}

% Take header off of first page
\thispagestyle{empty}

% Put in first page info (top of page)
\firstpageinfo

% Put in title for the paper
\problemlist

Show that, if $B \in \sigma(\sA)$, then there exists a countable subclass $\sA_B$ of $\sA$ such that $B \in \sigma(\sA_B)$.
\begin{proof}
Implicit in the statement of the problem is that $\sA$ is a subset of the power set of $\Omega$.

If $\sA$ is countable, then take $\sA_B = \sA$. So suppose $\sA$ is uncountable. Define
\[
\cZ = \{ \zeta : \zeta \subset \sA
\text{ and $\zeta$ is countable}
\}
\]
Note $\cZ$ is nonempty since $\sA$ is nonempty.
Let
\[
\cF = \bigcup_{\zeta \in \cZ} \sigma(\zeta)
\]
Then $\cF$ is union of all $\sigma$-algebras generated by an element of $\cZ$. The goal is to show that $\cF = \sigma(\sA)$.

Given $\zeta \in \cZ$, it follows that $\Omega \in \sigma(\zeta)$ by properties of $\sigma$-algebra. Therefore $\Omega \in \cF$ because $\sigma(\zeta) \subset \cF$.

Let $A \in \cF$. Then exists a $\zeta_0 \in \cZ$ such that $A \in \sigma(\zeta_0)$. By properties of $\sigma$-algebra, $A^c \in \sigma(\zeta_0)$. Thus
\[
A^c \in \sigma(\zeta_0) \subset \cF
\]

Suppose $A_1, A_2, \cdots$ is a sequence of elements in $\cF$. Then there exist $\zeta_1, \zeta_2, \cdots$ in $\cZ$ such that $A_i \in \sigma(\zeta_i)$ for all $i$. Since the countable union of countable sets is itself countable, it follows that
$
\cup_{j=1}^\infty \zeta_j \in \cZ
$.
For all $i$,
\[
\zeta_i 
\subset 
\bigcup_{j=1}^\infty \zeta_j
\subset
\sigma
\left(
\bigcup_{j=1}^\infty \zeta_j
\right)
\subset
\cF
\]
Therefore, since $\sigma(\zeta_i)$ is a subset of all $\sigma$-algebras that contain $\zeta_i$, it follows that 
\[
\sigma(\zeta_i) 
\subset 
\sigma
\left(
\bigcup_{j=1}^\infty \zeta_j
\right)
\]
for all $i$. Hence, 
%there exists a $\sigma$-field such that $A_i$ is an element of it for all $i$. Namely,
\[
A_i 
\in 
\sigma(\zeta_i)
\subset
\sigma
\left(
\bigcup_{j=1}^\infty \zeta_j
\right)
\]
for all $i$. By properties of $\sigma$-field, 
\[
\bigcup_{i=1}^\infty 
A_i
\in
\sigma
\left(
\bigcup_{j=1}^\infty \zeta_j
\right)
\subset
\cF
\]
%In other words 
%\[
%\bigcup_{i=1}^\infty
%\sigma(\zeta_i) 
%\subset 
%\sigma
%\left(
%\bigcup_{j=1}^\infty \zeta_j
%\right)
%\subset
%\cF
%\]
%
%\[
%\bigcup_{h=1}^\infty
%A_h
%\in
%\bigcup_{i=1}^\infty
%\sigma(\zeta_i) 
%\subset 
%\sigma
%\left(
%\bigcup_{j=1}^\infty \zeta_j
%\right)
%\subset
%\cF
%\]
Thus $\cF$ is closed under countable union, and it has been demonstrated that $\cF$ is a $\sigma$-algebra.

Suppose $A \in \sA$. Then $\{A\} \in \cZ$ and it follows that $A \in \sigma(\{A\}) \subset \cF$.  Thus $\cF$ is a $\sigma$-field that contains $\sA$. Hence $\sigma(\sA) \subset \cF$. Suppose $F \in \cF$. Then there exists $\zeta \in \cZ$ such that $F \in \sigma(\zeta)$. Since $\zeta \subset \sA \subset \sigma(\sA)$, it follows that $\sigma(\zeta) \subset \sigma(\sA)$. 
Therefore, $F \in \sigma(\zeta) \subset \sigma(\sA)$ and $\cF \subset \sigma(\sA)$. Since the set inclusion has been shown in both directions,
\[
\cF = \sigma(\sA)
\]
As shown in the paragraph above, given any $B \in \cF = \sigma(\sA)$, there exists $\zeta \in \cZ$, some subset of $\sA$, such that $B \in \sigma(\zeta)$. Define $\sA_B := \zeta$, and it is clear that $\sA_B$ is a countable subclass of $\sA$.
\end{proof}

\section*{References}
\begin{itemize}
\item
Notes on the definition of countable \url{http://en.wikipedia.org/wiki/Countable_set}. In this problem, ``countable'' means ``has the same cardinality as a subset of the set of natural numbers.''
\item
\url{http://math.stackexchange.com/questions/297942/reference-request-set-theory-of-sigma-algebras}
\item
\url{
http://math.stackexchange.com/questions/344784/show-that-for-any-a-in-sigma-vartheta-then-a-in-sigma-xi-k-where}
\item
\url{http://math.stackexchange.com/questions/61617/sigma-algebras}
\item
\url
{http://math.stackexchange.com/questions/496837/you-only-need-countable-many-sets-each-time-in-a-generated-sigma-algebra-rig}
\end{itemize}
\end{document}