
% /**
%  * A template for homework files in math classes. The 
%  * packages and newcommands are a good starting point.
%  *
%  * Author: James K. Pringle
%  * E-mail: jameskpringle@gmail.com
%  * Last Changed: 5 September 2013
%  *
%  * "LaTeX countains the increasing union of MS Word"
%  */
%~~~~~~~~~~~~~~~~~~~~~~~~~~~~~~~~~~~~~~~~~~~~~~~~~~~~~~~~~%
%%%%%%%%%%%%%%%%%%%%%%%%%%%%%%%%%%%%%%%%%%%%%%%%%%%%%%%%%%%
%                                                         %
%                        PAGE SETUP                       %
%                                                         %
%%%%%%%%%%%%%%%%%%%%%%%%%%%%%%%%%%%%%%%%%%%%%%%%%%%%%%%%%%%
\documentclass[letterpaper, 12pt]{article}

% 1in margins all the way around
\usepackage[margin=1in]{geometry}

% Sets \parindent to 0 and \parskip to stretchable.
\usepackage{parskip}
% Use for bigger spaces between paragraphs.
%\parskip=1.5\baselineskip

% Set headers and footers
\usepackage{fancyhdr}
\pagestyle{fancy}
% Header
\renewcommand{\headrulewidth}{0.4pt}
\lhead{\textsc{\mathclass}}
\chead{\textsc{\today}}
\rhead{\textsc{\mynamehdr}}
% Footer
\renewcommand{\footrulewidth}{0.4pt}
\lfoot{}
\cfoot{\thepage}
\rfoot{}

% Make the space between lines slightly more generous 
% than normal single spacing, but compensate so that the 
% spacing between rows of matrices still looks normal.  
% Note that 1.1=1/.9090909...
\renewcommand{\baselinestretch}{1.1}
\renewcommand{\arraystretch}{.91}

%%%%%%%%%%%%%%%%%%%%%%%%%%%%%%%%%%%%%%%%%%%%%%%%%%%%%%%%%%%
%                                                         %
%                      USEFUL PACKAGES                    %
%                                                         %
%%%%%%%%%%%%%%%%%%%%%%%%%%%%%%%%%%%%%%%%%%%%%%%%%%%%%%%%%%%

% The classic three
\usepackage{amsmath,amsthm,amssymb}

% Define \newtheorem for use
% No numbers, labeled 'Theorem'
\newtheorem*{nthm}{Theorem}

% Not sure what this is for
\usepackage{amsfonts}

% Fancy script font
\usepackage{mathrsfs}

% Makes enumerate environment much easier to customize
% by specifying the counter
\usepackage{enumerate}

% Color
\usepackage{color}
\usepackage[usenames,dvipsnames,svgnames,table]{xcolor}

% URL links
\usepackage{hyperref}

% For inserting graphics and images
\usepackage{graphicx}
\usepackage{float}
\usepackage[footnotesize]{caption}



%%%%%%%%%%%%%%%%%%%%%%%%%%%%%%%%%%%%%%%%%%%%%%%%%%%%%%%%%%%
%                                                         %
%                   USER-DEFINED COMMANDS                 %
%                                                         %
%%%%%%%%%%%%%%%%%%%%%%%%%%%%%%%%%%%%%%%%%%%%%%%%%%%%%%%%%%%

% Make a hyperlink with colored text
\newcommand{\hrefcolor}[3]{\href{#1}{\textcolor{#3}{#2}}}

% Make a hyperlink with gray text
\newcommand{\hrefgray}[2]{\hrefcolor{#1}{#2}{Gray}}

% Make the header for the first page
\newcommand{\firstpageinfo}{
\textsf{
\begin{flushleft}
\sc \myname \\
\normalfont \mathclass \\
\professorname \\
\assignmentnumber \\
\thedate
\end{flushleft}
} \bigskip
}

% Make problem list for "title" of page
\newcommand{\problemlist}{ 
\begin{center}
\textbf{\Large \textsf{\assignmentnumber}}\\
\textit{\textsf{\problemset}}
\end{center}
\bigskip
}

%~~~~~~~~~~~~~~~~~~~~~~~~~~~~~~~~~~~~~~~~~~~~~~~~~~~~~~~~~%
%                                                         %
%               LETTERS, FUNCTIONS, AND TEXT              %
%                                                         %
%~~~~~~~~~~~~~~~~~~~~~~~~~~~~~~~~~~~~~~~~~~~~~~~~~~~~~~~~~%

% A
\newcommand{\cA}{\mathcal{A}}
\newcommand{\sA}{\mathscr{A}}
\renewcommand{\aa}{\;\text{a.a.}}
\renewcommand{\ae}{\;\text{a.e.}}
% B
\newcommand{\B}{\mathscr{B}}
\newcommand{\cB}{\mathcal{B}}
% C
\newcommand{\cC}{\mathcal{C}}
\newcommand{\cov}{\text{cov}}
% E
\newcommand{\E}{\mathbb{E}}
% F
\newcommand{\sF}{\mathscr{F}}
\newcommand{\cF}{\mathcal{F}}
\newcommand{\Ft}{F^\sim}
% G
\newcommand{\cG}{\mathcal{G}}
\newcommand{\sG}{\mathscr{G}}
% I
\newcommand{\io}{\;\text{i.o.}}
% N
\newcommand{\N}{\mathbb{N}}
% P
\newcommand{\cP}{\mathcal{P}}
\newcommand{\sP}{\mathscr{P}}
\newcommand{\pr}{\text{pr}}
% Q
\newcommand{\Q}{\mathbb{Q}}
% R
\newcommand{\R}{\mathbb{R}}
\newcommand{\bR}{\mathbf{R}}
\newcommand{\cR}{\mathcal{R}}
% S
\newcommand{\cS}{\mathcal{S}}
% U
\newcommand{\cU}{\mathcal{U}}
% V
\newcommand{\var}{\text{var}}
% Z
\newcommand{\Z}{\mathbb{Z}}
% Punctuation
\newcommand{\sbs}{\;|\;} % space bar space
% Math
\newcommand{\imii}{\int_{-\infty}^\infty}
\newcommand{\pion}{\prod_{i=1}^n}
\newcommand{\pioI}{\prod_{i=1}^I}
\newcommand{\pjon}{\prod_{j=1}^n}
\newcommand{\pjoJ}{\prod_{j=1}^J}
\newcommand{\pkon}{\prod_{k=1}^n}
\newcommand{\pkoK}{\prod_{k=1}^K}
\newcommand{\sion}{\sum_{i=1}^n}
\newcommand{\sioI}{\sum_{i=1}^I}
\newcommand{\sjon}{\sum_{j=1}^n}
\newcommand{\sjoJ}{\sum_{j=1}^J}
\newcommand{\skon}{\sum_{k=1}^n}
\newcommand{\skoK}{\sum_{k=1}^K}
\newcommand{\sioi}{\sum_{i=1}^\infty}
\newcommand{\sjoi}{\sum_{j=1}^\infty}
\newcommand{\skoi}{\sum_{k=1}^\infty}
\newcommand{\sio}{\sum_{i=1}}
\newcommand{\sjo}{\sum_{j=1}}
\newcommand{\sko}{\sum_{k=1}}
% Typography
\newcommand{\scb}[1]{\textsc{\textbf{#1}}}

%~~~~~~~~~~~~~~~~~~~~~~~~~~~~~~~~~~~~~~~~~~~~~~~~~~~~~~~~~%
%                                                         %
%            CHANGE THESE BASED ON THE PAPER              %
%                                                         %
%~~~~~~~~~~~~~~~~~~~~~~~~~~~~~~~~~~~~~~~~~~~~~~~~~~~~~~~~~%

% Constants for fancy header and first page info
\newcommand{\mynamehdr}{\hrefgray{http://biostat.jhsph.edu/~jpringle/}{\myname}}
\newcommand{\mathclass}{550.621 Probability}
\newcommand{\myname}{James K. Pringle}
\newcommand{\professorname}{Dr. Jim Fill}
\newcommand{\assignmentnumber}{Assignment 4}
\newcommand{\thedate}{\today}
\newcommand{\problemset}{Chung Exercise 7.2.4}

%%%%%%%%%%%%%%%%%%%%%%%%%%%%%%%%%%%%%%%%%%%%%%%%%%%%%%%%%%%
%                                                         %
%                      BEGIN DOCUMENT                     %
%                                                         %
%%%%%%%%%%%%%%%%%%%%%%%%%%%%%%%%%%%%%%%%%%%%%%%%%%%%%%%%%%%
\begin{document}

% Take header off of first page
\thispagestyle{empty}

% Put in first page info (top of page)
\firstpageinfo

% Put in title for the paper
\problemlist


%%%%%%%%%%%%%%%%%%%%%%%%%%%%%%%%%%%%%%%%%%%%%%%%%%%%%%%%%%%
%                                                         %
%                     Start Problem 1                     %
%                                                         %
%%%%%%%%%%%%%%%%%%%%%%%%%%%%%%%%%%%%%%%%%%%%%%%%%%%%%%%%%%%
\section*{Chung Exercise 7.2.4}
Prove the sufficiency part of Theorem 7.2.1 without using Theorem 7.1.2, but by elaborating the proof of the latter.

\section*{Chung Theorem 7.2.1}
Assume $\sigma_{nj}^2 < \infty$ for each $n$ and $j$ and the reduction hypotheses
\begin{align}
\sum_{j=1}^{k_n} \sigma^2(X_{nj}) &= 1
\label{vo}
\\
E(X_{nj}) &= 0
\label{ez}
\end{align}
of Sec 7.1. In order that as $n \to \infty$ the two conclusions below both hold
\begin{enumerate}
\item[(I)]
$S_n$ converges in dist. to $\Phi$ (standard normal)
\item[(II)]
the double array (2) of Sec. 7.1 is holospoudic,
\end{enumerate}
it is necessary and sufficient that for each $\eta >0$, we have
\begin{equation}
\sum_{j=1}^{k_n} \int_{|x| > \eta} x^2 dF_{nj}(x) \to 0
\label{lc}
\end{equation}

\section*{Assumptions}
By way of notation, let $F_{nj}(x)$ be the d.f. of $X_{nj}$.
Assume the hypotheses of \textbf{Chung Theorem 7.2.1} and assume \eqref{lc}. 
Note by \eqref{vo}
\[
1 = \sum_{j=1}^{k_n} \sigma^2(X_{nj})
=
\sum_{j=1}^{k_n}
\int
x^2
dF_{nj}(x)
=
\sum_{j=1}^{k_n}
\left(
\int_{|x| \leq \eta}
+
\int_{|x| > \eta}
\right)
x^2
dF_{nj}(x)
\]
so that
\[
0 
=
\lim_{n\to\infty}
\sum_{j=1}^{k_n} \int_{|x| > \eta} x^2 dF_{nj}(x)
=
1 
-
\lim_{n\to\infty}
\sum_{j=1}^{k_n} \int_{|x| \leq \eta} x^2 dF_{nj}(x)
\]
Thus
\begin{equation}
\lim_{n\to\infty}
\sum_{j=1}^{k_n} \int_{|x| \leq \eta} x^2 dF_{nj}(x)
= 
1
\label{flip}
\end{equation}

\section*{Discussion}
The proof examines the convergence of ch.f.'s of the rows of the double array. By hypothesis, $S_n$ is the sum of the $k_n$ independent random variables in the $n$-th row. By independence, the ch.f. of $S_n$ is the product of the ch.f.'s of the random variables of that row. Define $f_n$ to be the ch.f. of $S_n$.
Define $f_{nj}$ to be the ch.f. of $X_{nj}$
\[
f_n(t) = f_{S_n}(t) = \prod_{j=1}^{k_n} f_{X_{nj}}(t)
=
\prod_{j=1}^{k_n} f_{nj}(t)
\]
By \textbf{Chung Theorem 6.3.2} it is sufficient to show for all $t \in \R$ that $f_n(t)$ converges to the ch.f. of standard normal, $e^{-\frac{t^2}{2}}$ in order that $S_n$ converges in distribution to $\Phi$.

Next follow some lemmas that are required for the proof.

\subsection*{Lemma 1}
Let $\{ \theta_{nj}, 1\leq j \leq k_n, 1 \leq n  \}$ be a double array of complex numbers satisfying the following conditions as $n \to \infty$
\begin{enumerate}[(i)]
\item
$\max_{1 \leq j \leq k_n} |\theta_{nj}| \to 0$
\item
$\sum_{j=1}^{k_n} |\theta_{nj}| \leq M < \infty$, where $M$ does not depend on $n$
\item
$\sum_{j=1}^{k_n} \theta_{nj} \to \theta$, where $\theta$ is a (finite) complex number.
\end{enumerate}
Then we have 
\begin{equation*}
\prod_{j=1}^{k_n} (1 + \theta_{nj}) \to e^{\theta}
\end{equation*}

\subsection*{Lemma 2}
According to Billingsley $26.4_1$
\[
|e^{ix} - (1 + ix) | \leq \min \{   \frac{1}{2} x^2, 2|x| \} 
\]

\subsection*{Lemma 3}
According to Billingsley $26.4_2$
\[
|e^{ix} - (1 + ix -\frac{1}{2}x^2 ) | \leq \min \{   \frac{1}{6} |x|^3, x^2 \} 
\]

\section*{Proof of the Main Result}

Fix $t \in \R$. Notice
\[
f_n(t) = \prod_{j=1}^{k_n} f_{nj}(t) = \prod_{j=1}^{k_n} 1 + (f_{nj}(t) - 1)
\]
To apply $\textbf{Lemma 1}$, let $\theta_{nj} = f_{nj}(t) - 1$.

\subsubsection*{Condition (i)}
Let $\eta >0$. Calculating,
\begin{align*}
\max_{1 \leq j \leq k_n} |\theta_{nj}|
&=
\max_{1 \leq j \leq k_n} |f_{nj}(t) - 1|
\\
&=
\max_{1 \leq j \leq k_n} |f_{nj}(t) - 1 - it E(X_{nj})|
\quad
\text{by \eqref{ez}}
\\
&=
\max_{1 \leq j \leq k_n} 
\lvert
\int (e^{itx} - 1 - itx) dF_{nj}(x)
\rvert
\\
&\leq
\max_{1 \leq j \leq k_n}
\int |e^{itx} - 1 -itx| 
dF_{nj}(x)
\quad
\text{by modulus inequality}
\\
&\leq
\max_{1 \leq j \leq k_n}
\int
\frac{1}{2}
(tx)^2
dF_{nj}(x)
\quad
\text{by \textbf{Lemma 2}}
\\
&=
\max_{1 \leq j \leq k_n}
\frac{1}{2} t^2 
\left(
\int_{|x| \leq \eta}
x^2
dF_{nj}(x)
+
\int_{|x| > \eta}
x^2
dF_{nj}(x)
\right)
\\
&\leq
\frac{1}{2} t^2 \eta^2
+
\frac{1}{2}t^2
\max_{1 \leq j \leq k_n}
\int_{|x| > \eta}
x^2
dF_{nj}(x)
\\
&\leq
\frac{1}{2} t^2 \eta^2
+
\frac{1}{2} t^2 
\sum_{j=1}^{k_n}
\int_{|x| > \eta}
x^2
dF_{nj}(x)
\end{align*}
since the max of a set of positive numbers is less than or equal to the sum of all of them.
Thus, by \eqref{lc},
\[
0 
\leq
\lim_{n \to \infty} 
\left(
\max_{1 \leq j \leq k_n} |\theta_{nj}|
\right)
\leq 
\frac{1}{2} t^2 \eta^2
+
\frac{1}{2}t^2
\lim_{n \to \infty}
\left(
\sum_{j=1}^{k_n}
\int_{|x| > \eta}
x^2
dF_{nj}(x)
\right)
=
\frac{1}{2} t^2 \eta^2
\]
Since there is no dependence on $\eta$ in the second term above, and since $\eta$ is arbitrary and positive,
\[
0
\leq
\lim_{n \to \infty} 
\left(
\max_{1 \leq j \leq k_n} |\theta_{nj}|
\right)
%=
%\lim_{\eta \to 0}
%\left(
%\lim_{n \to \infty} 
%\left(
%\max_{1 \leq j \leq k_n} |\theta_{nj}|
%\right)
%\right)
\leq
\lim_{\eta \to 0}
\left(
\frac{1}{2} t^2 \eta^2
\right)
=
0
\]
Hence \textbf{Condition (i)} holds.
\subsubsection*{Condition (ii)}
Calculating,
\begin{align*}
\sum_{j=1}^{k_n} |\theta_{nj}|
&=
\sum_{j=1}^{k_n} |f_{nj}(t) - 1|
\\
&\leq
\sum_{j=1}^{k_n}
\int
\frac{1}{2} (tx)^2
dF_{nj}(x)
\quad
\text{by same calculations as for \textbf{Condition (i)}}
\\
&=
\frac{1}{2}t^2 
\sum_{j=1}^{k_n}
\int
x^2
dF_{nj}(x)
\\
&=
\frac{1}{2}t^2 
\sum_{j=1}^{k_n}
\sigma^2(X_{nj})
\\
&=
\frac{1}{2}t^2 
\quad
\text{by \eqref{vo}.}
\end{align*}
Since this bound does not depend on $n$ and it is finite, \textbf{Condition (ii)} holds.

\subsubsection*{Condition (iii)}
Let $\eta >0$.
It is claimed that $\theta = -\frac{t^2}{2}$, so that 
\[
\sum_{j=1}^{k_n} \theta_{nj} \to \theta.
\]
Calculating,
\begin{align*}
\left|
\left(
\sum_{j=1}^{k_n} \theta_{nj} 
\right)
- \theta
\right|
&=
\left|
\sum_{j=1}^{k_n} 
\left(
f_{nj}(t) 
- 1 
\right)
+ \frac{t^2}{2}
\right|
\\
&=
\left|
\sum_{j=1}^{k_n} 
\left(
f_{nj}(t) 
- 1
\right)
- 
\frac{(it)^2}{2}
\left(\sum_{j=1}^{k_n} \sigma^2(X_{nj})\right)
\right|
\quad
\text{by \eqref{vo}}
\\
&=
\left|
\sum_{j=1}^{k_n} 
\left(
f_{nj}(t) 
- 1 -it E(X_{nj})
\right)
-
\frac{(it)^2}{2}
\left(\sum_{j=1}^{k_n} E(X_{nj}^2)\right)
\right|
\quad
\text{by \eqref{ez}}
\\
&=
\left|
\sum_{j=1}^{k_n} 
\left(
f_{nj}(t) 
- 1 
-
it E(X_{nj})
-
\frac{(it)^2}{2}E(X_{nj}^2)
\right)
\right|
\\
&=
\left|
\sum_{j=1}^{k_n} 
\left(
\int
e^{itx}
-1
-itx
+\frac{(tx)^2}{2}
dF_{nj}(x)
\right)
\right|
\\
&\leq
\sum_{j=1}^{k_n} 
\int
\left|
e^{itx}
-1
-itx
+\frac{(tx)^2}{2}
\right|
dF_{nj}(x)
\quad
\text{by triangle and modulus inequalities}
\\
&=
\sum_{j=1}^{k_n} 
\left(
\int_{|x| \leq \eta}
\left|
e^{itx}
-1
-itx
+\frac{(tx)^2}{2}
\right|
dF_{nj}(x)
+
\int_{|x| > \eta}
\left|
e^{itx}
-1
-itx
+\frac{(tx)^2}{2}
\right|
dF_{nj}(x)
\right)
\\
&\leq
\sum_{j=1}^{k_n} 
\left(
\int_{|x| \leq \eta}
\frac{|x|^3}{6}
dF_{nj}(x)
+
%\sum_{j=1}^{k_n} 
\int_{|x| > \eta}
x^2
dF_{nj}(x)
\right)
\quad
\text{by \textbf{Lemma 3}}
\\
&\leq
\frac{\eta}{6}
\sum_{j=1}^{k_n} 
\int_{|x| \leq \eta}
x^2
dF_{nj}(x)
+
\sum_{j=1}^{k_n} 
\int_{|x| > \eta}
x^2
dF_{nj}(x)
\end{align*}
Taking the limit of this nonegative quantity as $n$ tends to infinity,
\begin{align*}
0 
&\leq
\lim_{n \to \infty}
\left|
\left(
\sum_{j=1}^{k_n} \theta_{nj} 
\right)
- \theta
\right|
\\
&\leq
\lim_{n \to \infty}
\left(
\frac{\eta}{6}
\sum_{j=1}^{k_n} 
\int_{|x| \leq \eta}
x^2
dF_{nj}(x)
+
\sum_{j=1}^{k_n} 
\int_{|x| > \eta}
x^2
dF_{nj}(x)
\right)
\\
&=
\frac{\eta}{6}
\lim_{n \to \infty}
\left(
\sum_{j=1}^{k_n} 
\int_{|x| \leq \eta}
x^2
dF_{nj}(x)
\right)
+
\lim_{n \to \infty}
\left(
\sum_{j=1}^{k_n} 
\int_{|x| > \eta}
x^2
dF_{nj}(x)
\right)
\\
&=
\frac{\eta}{6}
\quad
\text{by \eqref{flip} and \eqref{lc}}
\end{align*}
Since $\eta$ is arbitrary and positive, by squeezing,
\[
0 
\leq 
\lim_{n \to \infty}
\left|
\left(
\sum_{j=1}^{k_n} \theta_{nj} 
\right)
- \theta
\right|
\leq 
\lim_{\eta \to 0}
\frac{\eta}{6}
=
0
\]
and it follows that 
\[
\lim_{n \to \infty}
\left|
\left(
\sum_{j=1}^{k_n} \theta_{nj} 
\right)
- \theta
\right| = 0
\]
Hence,
$\sum_{j=1}^{k_n} \theta_{nj} $ converges to $\theta=-\frac{t^2}{2}$ as $n$ tends to infinity. \textbf{Condition (iii)} holds. Therefore by \textbf{Lemma 1}, 
\[
\lim_{n \to \infty}
f_{S_n}(t) 
=
\lim_{n \to \infty}
f_n(t) 
=
\prod_{j=1}^{k_n} 1 + (f_{nj}(t) - 1)
=
\prod_{j=1}^{k_n} 1 + \theta_{nj}
=
e^{\theta}
=
e^{-\frac{t^2}{2}}
\]

Note $t \in \R$ is arbitrary. Thus for all $t \in \R$ it is the case that $f_{S_n}(t)$ converges to the ch.f. of standard normal, $e^{-\frac{t^2}{2}}$, and by \textbf{Chung Theorem 6.3.2} $S_n$ converges in distribution to $\Phi$. Thus (I).

Condition (II), holospoudicity, follows from \textbf{Chung Theorem 7.1.1} since \textbf{Condition (i)} holds. Alternatively, the proof of holospoudicity copies over from Chung since it does not rely on \textbf{Chung Theorem 7.1.2}. Therefore, \eqref{lc} is sufficient for (I) and (II).

\section*{Acknowledgements}
I worked on this problem with Elizabeth, David, and Vivek.



\end{document}