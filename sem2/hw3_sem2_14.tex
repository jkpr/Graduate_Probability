% /**
%  * A template for homework files in math classes. The 
%  * packages and newcommands are a good starting point.
%  *
%  * Author: James K. Pringle
%  * E-mail: jameskpringle@gmail.com
%  * Last Changed: 5 September 2013
%  *
%  * "LaTeX countains the increasing union of MS Word"
%  */
%~~~~~~~~~~~~~~~~~~~~~~~~~~~~~~~~~~~~~~~~~~~~~~~~~~~~~~~~~%
%%%%%%%%%%%%%%%%%%%%%%%%%%%%%%%%%%%%%%%%%%%%%%%%%%%%%%%%%%%
%                                                         %
%                        PAGE SETUP                       %
%                                                         %
%%%%%%%%%%%%%%%%%%%%%%%%%%%%%%%%%%%%%%%%%%%%%%%%%%%%%%%%%%%
\documentclass[letterpaper, 12pt]{article}

% 1in margins all the way around
\usepackage[margin=1in]{geometry}

% Sets \parindent to 0 and \parskip to stretchable.
\usepackage{parskip}
% Use for bigger spaces between paragraphs.
%\parskip=1.5\baselineskip

% Set headers and footers
\usepackage{fancyhdr}
\pagestyle{fancy}
% Header
\renewcommand{\headrulewidth}{0.4pt}
\lhead{\textsc{\mathclass}}
\chead{\textsc{\today}}
\rhead{\textsc{\mynamehdr}}
% Footer
\renewcommand{\footrulewidth}{0.4pt}
\lfoot{}
\cfoot{\thepage}
\rfoot{}

% Make the space between lines slightly more generous 
% than normal single spacing, but compensate so that the 
% spacing between rows of matrices still looks normal.  
% Note that 1.1=1/.9090909...
\renewcommand{\baselinestretch}{1.1}
\renewcommand{\arraystretch}{.91}

%%%%%%%%%%%%%%%%%%%%%%%%%%%%%%%%%%%%%%%%%%%%%%%%%%%%%%%%%%%
%                                                         %
%                      USEFUL PACKAGES                    %
%                                                         %
%%%%%%%%%%%%%%%%%%%%%%%%%%%%%%%%%%%%%%%%%%%%%%%%%%%%%%%%%%%

% The classic three
\usepackage{amsmath,amsthm,amssymb}

% Define \newtheorem for use
% No numbers, labeled 'Theorem'
\newtheorem*{nthm}{Theorem}

% Not sure what this is for
\usepackage{amsfonts}

% Fancy script font
\usepackage{mathrsfs}

% Makes enumerate environment much easier to customize
% by specifying the counter
\usepackage{enumerate}

% Color
\usepackage{color}
\usepackage[usenames,dvipsnames,svgnames,table]{xcolor}

% URL links
\usepackage{hyperref}

% For inserting graphics and images
\usepackage{graphicx}
\usepackage{float}
\usepackage[footnotesize]{caption}



%%%%%%%%%%%%%%%%%%%%%%%%%%%%%%%%%%%%%%%%%%%%%%%%%%%%%%%%%%%
%                                                         %
%                   USER-DEFINED COMMANDS                 %
%                                                         %
%%%%%%%%%%%%%%%%%%%%%%%%%%%%%%%%%%%%%%%%%%%%%%%%%%%%%%%%%%%

% Make a hyperlink with colored text
\newcommand{\hrefcolor}[3]{\href{#1}{\textcolor{#3}{#2}}}

% Make a hyperlink with gray text
\newcommand{\hrefgray}[2]{\hrefcolor{#1}{#2}{Gray}}

% Make the header for the first page
\newcommand{\firstpageinfo}{
\textsf{
\begin{flushleft}
\sc \myname \\
\normalfont \mathclass \\
\professorname \\
\assignmentnumber \\
\thedate
\end{flushleft}
} \bigskip
}

% Make problem list for "title" of page
\newcommand{\problemlist}{ 
\begin{center}
\textbf{\Large \textsf{\assignmentnumber}}\\
\textit{\textsf{\problemset}}
\end{center}
\bigskip
}

%~~~~~~~~~~~~~~~~~~~~~~~~~~~~~~~~~~~~~~~~~~~~~~~~~~~~~~~~~%
%                                                         %
%               LETTERS, FUNCTIONS, AND TEXT              %
%                                                         %
%~~~~~~~~~~~~~~~~~~~~~~~~~~~~~~~~~~~~~~~~~~~~~~~~~~~~~~~~~%

% A
\newcommand{\cA}{\mathcal{A}}
\newcommand{\sA}{\mathscr{A}}
\renewcommand{\aa}{\;\text{a.a.}}
\renewcommand{\ae}{\;\text{a.e.}}
% B
\newcommand{\B}{\mathscr{B}}
\newcommand{\cB}{\mathcal{B}}
% C
\newcommand{\cC}{\mathcal{C}}
\newcommand{\cov}{\text{cov}}
% E
\newcommand{\E}{\mathbb{E}}
% F
\newcommand{\fF}{\mathfrak{F}}
\newcommand{\sF}{\mathscr{F}}
\newcommand{\cF}{\mathcal{F}}
\newcommand{\Ft}{F^\sim}
% G
\newcommand{\cG}{\mathcal{G}}
\newcommand{\sG}{\mathscr{G}}
% I
\newcommand{\io}{\;\text{i.o.}}
% N
\newcommand{\N}{\mathbb{N}}
% P
\newcommand{\cP}{\mathcal{P}}
\newcommand{\sP}{\mathscr{P}}
\newcommand{\pr}{\text{pr}}
% Q
\newcommand{\Q}{\mathbb{Q}}
% R
\newcommand{\R}{\mathbb{R}}
\newcommand{\bR}{\mathbf{R}}
\newcommand{\cR}{\mathcal{R}}
% S
\newcommand{\cS}{\mathcal{S}}
% U
\newcommand{\cU}{\mathcal{U}}
% V
\newcommand{\var}{\text{var}}
% Z
\newcommand{\Z}{\mathbb{Z}}
% Punctuation
\newcommand{\sbs}{\;|\;} % space bar space
% Math
\newcommand{\imii}{\int_{-\infty}^\infty}
\newcommand{\pion}{\prod_{i=1}^n}
\newcommand{\pioI}{\prod_{i=1}^I}
\newcommand{\pjon}{\prod_{j=1}^n}
\newcommand{\pjoJ}{\prod_{j=1}^J}
\newcommand{\pkon}{\prod_{k=1}^n}
\newcommand{\pkoK}{\prod_{k=1}^K}
\newcommand{\sion}{\sum_{i=1}^n}
\newcommand{\sioI}{\sum_{i=1}^I}
\newcommand{\sjon}{\sum_{j=1}^n}
\newcommand{\sjoJ}{\sum_{j=1}^J}
\newcommand{\skon}{\sum_{k=1}^n}
\newcommand{\skoK}{\sum_{k=1}^K}
\newcommand{\sioi}{\sum_{i=1}^\infty}
\newcommand{\sjoi}{\sum_{j=1}^\infty}
\newcommand{\skoi}{\sum_{k=1}^\infty}
\newcommand{\sio}{\sum_{i=1}}
\newcommand{\sjo}{\sum_{j=1}}
\newcommand{\sko}{\sum_{k=1}}
% Typography
\newcommand{\scb}[1]{\textsc{\textbf{#1}}}

%~~~~~~~~~~~~~~~~~~~~~~~~~~~~~~~~~~~~~~~~~~~~~~~~~~~~~~~~~%
%                                                         %
%            CHANGE THESE BASED ON THE PAPER              %
%                                                         %
%~~~~~~~~~~~~~~~~~~~~~~~~~~~~~~~~~~~~~~~~~~~~~~~~~~~~~~~~~%

% Constants for fancy header and first page info
\newcommand{\mynamehdr}{\hrefgray{http://biostat.jhsph.edu/~jpringle/}{\myname}}
\newcommand{\mathclass}{550.621 Probability}
\newcommand{\myname}{James K. Pringle}
\newcommand{\professorname}{Dr. Jim Fill}
\newcommand{\assignmentnumber}{Assignment 3}
\newcommand{\thedate}{\today}
\newcommand{\problemset}{Chung 6.4.19}

%%%%%%%%%%%%%%%%%%%%%%%%%%%%%%%%%%%%%%%%%%%%%%%%%%%%%%%%%%%
%                                                         %
%                      BEGIN DOCUMENT                     %
%                                                         %
%%%%%%%%%%%%%%%%%%%%%%%%%%%%%%%%%%%%%%%%%%%%%%%%%%%%%%%%%%%
\begin{document}

% Take header off of first page
\thispagestyle{empty}

% Put in first page info (top of page)
\firstpageinfo

% Put in title for the paper
\problemlist
%\section*{Notation}
%The symbol $\Rightarrow$ denotes \textit{weak convergence}. If for d.f.'s $F_n$ and $F$
%\[
%F_n \Rightarrow F
%\]
%then $F_n$ converges weakly to $F$.
%
%If for measures $\mu_n$ and $\mu$
%\[
%\mu_n \Rightarrow \mu
%\]
%then $\mu_n$ converges weakly to $\mu$.

\section*{Problem}
The problem is to provide a solution to Chung Exercise 6.4.19, which is:
\begin{quote}
Reformulate Exercise 18 in terms of d.f.'s and deduce the following consequence. Let $F_n$ be a sequence of d.f.'s $a_n$, $a_n'$ real constants, $b_n > 0$, $b_n' > 0$. If
\[
F_n(b_n x + a_n) \xrightarrow{v} F(x)
\quad
\text{and}
\quad
F_n(b_n' x + a_n') \xrightarrow{v} F(x),
\]
where $F$ is a nondegenerate d.f., then 
\[
\frac{b_n}{b_n'} \to 1
\quad
\text{and}
\quad
\frac{a_n - a_n'}{b_n} \to 0
\]
\end{quote}
First, \textbf{Chung Exercise 6.4.18} is stated and the reformulation is presented. Following that, the consequence is deduced. 

\section*{Chung Exercise 6.4.18}
Let $f_n$ be ch.f.'s.
Let $f$ and $g$ be two nondegenerate ch.f.'s. Suppose that there exist real constants $a_n$ and $b_n > 0$ such that for every $t$:
\begin{equation}
f_n(t)\to f(t) \quad \text{and} \quad
e^{ita_n/b_n}
f_n
\left(
\frac{t}{b_n}
\right)
\to
g(t)
\label{eq1}
\end{equation}
Then $a_n \to a$, $b_n \to b$, where $a$ is finite, $0 < b<\infty$, and $g(t) = e^{ita/b} f(t/b)$.

\section*{Reformulation of 6.4.18 in terms of d.f.'s}

Let $\Phi_n$ be d.f.'s with ch.f.'s $f_n$.
Let $\Phi$ and $G$ be two nondegenerate d.f.'s with characteristic functions $f$ and $g$, respectively. Suppose that there exist real constants $\alpha_n$ and $\beta_n>0$ such that:
\begin{equation}
\Phi_n(x) \xrightarrow{v} \Phi(x) 
\quad \text{and} \quad
\Phi_n(\beta_nx - \alpha_n) 
\xrightarrow{v}
G(x)
\label{rew}
\end{equation}
Then $\alpha_n \to \alpha$, $\beta_n \to \beta$, where $\alpha$ is finite, $0 < \beta<\infty$, and $G(x) = \Phi(\beta x - \alpha)$.

\begin{proof}
Let $\Phi_n(x) \xrightarrow{v} \Phi(x)$. 
Then by definition, the corresponding p.m.'s on $\R$ converge weakly, and by the convergence theorem (\textbf{Chung Theoremm 6.3.1}), $f_n(t) \to f(t)$ for every $t$.

Let $\Phi_n(\beta_nx - \alpha_n) 
\xrightarrow{v}
G(x)$. Since $\beta_n > 0$, it follows that for all $n$, the function $T_n(x) = \beta_n x - \alpha_n$ is a continuous, increasing bijection. 
Therefore, $\Phi_n(T_n(x)) = \Phi_n(\beta_nx - \alpha_n)$ is a distribution function. 
The characteristic function of $\Phi_n(\beta_nx - \alpha_n)$ is 
\begin{align*}
\int e^{itx} d\Phi_n(\beta_nx - \alpha_n)
&=
\int e^{it\frac{x+\alpha_n}{\beta_n}} d\Phi_n(x)
\\
&=
e^{it\alpha_n/\beta_n}\int e^{i\frac{t}{\beta_n} x} d\Phi_n(x)
\\
&=
e^{it\alpha_n/\beta_n} f_n
\left(
\frac{t}{\beta_n}
\right)
\end{align*}
Since  $\Phi_n(\beta_nx - \alpha_n) 
\xrightarrow{v}
G(x)$, the corresponding p.m.'s on $\R$ converge weakly. Thus by the convergence theorem (\textbf{Chung Theorem 6.3.1}), the ch.f.'s converge for every $t$, i.e.
\[
e^{it\alpha_n/\beta_n} f_n
\left(
\frac{t}{\beta_n}
\right)
\to 
g(t)
\]
for every $t$.

The above shows that the set of assumptions in \eqref{rew} imply the set of assumptions in \eqref{eq1}. 
Therefore, given \eqref{rew}, it follows by \textbf{Chung Exercise 6.4.18} that $\alpha_n \to \alpha$, $\beta_n \to \beta$, where $\alpha$ is finite, $0 < \beta < \infty$, and $g(t) = e^{it\alpha/\beta}f(t/\beta)$. 
Additionally, by \textbf{Chung Theorem 6.3.2}, the sequence of d.f.'s $\{\Phi_n(\beta_nx - \alpha_n) \}_{n \geq 1}$ converges weakly to a d.f. $G$ with ch.f. $e^{it\alpha/\beta}f(t/\beta)$. 
Notice that $\Phi(\beta x -\alpha)$ has ch.f. $e^{it\alpha/\beta}f(t/\beta)$. 
Since any two d.f.'s with the same ch.f. are the same d.f. by \textbf{Chung Theorem 6.2.2}, it follows that $G(x) = \Phi(\beta x -\alpha)$.
\end{proof}

\section*{Deduction of the Consequence}
\begin{proof}
Let $F_n$ be a sequence of d.f.'s $a_n$, $a_n'$ real constants $b_n > 0$, $b_n' > 0$.
Let
\begin{equation*}
F_n(b_n x + a_n) \xrightarrow{v} F(x)
\quad
\text{and}
\quad
F_n(b_n'x + a_n') \xrightarrow{v} F(x)
\end{equation*}
where $F$ is a nondegenerate d.f. Since $b_n > 0$, for all $n$ it follows that $T_n(x) = b_n x + a_n$ is a continuous, increasing bijection. 
Therefore $\Phi_n(x) = F_n(T_n(x)) = F_n(b_n x + a_n)$ is a d.f. Using this new notation, $\Phi_n(x) \xrightarrow{v} F(x)$. Define
\begin{align*}
\beta_n &:= \frac{b_n'}{b_n}
\\
\alpha_n &:= -\frac{a_n' - a_n}{b_n}
\end{align*}
Notice $\beta_n$ is positive and $\alpha_n$ is some real number.
Calculating,
\begin{align*}
\Phi_n(\beta_n x - \alpha_n) 
&=
F_n(T_n(\beta_n x - \alpha_n))
\\
&=
F_n(b_n(\beta_n x - \alpha_n) + a_n)
\\
&=
F_n
\left(
b_n
\left(
\frac{b_n'}{b_n} x + \frac{a_n' - a_n}{b_n}
\right)
+ a_n
\right)
\\
&=
F(b_n' x + a_n')
\end{align*}
By hypothesis then, $\Phi_n(\beta_n x -\alpha_n) = F(b_n'x + a_n') \xrightarrow{v} F(x)$. 
%Notice both $b_n'>0$ and $\beta_n' >0$
By the \textbf{Reformulation of 6.4.18}, 
\[
\lim_{n \to \infty} \alpha_n = \alpha
\]
where $\alpha$ is finite, and 
\[
\lim_{n \to \infty} \beta_n = \beta
\]
where $0 < \beta < \infty$. Furthermore, as a result of the \textbf{Reformulation of 6.4.18}, 
the weak limit of $\{\Phi_n(\beta_n x -\alpha_n)\}_{n \geq 1}$ is $F(x) = F(\beta x - \alpha)$. Since this equality is true for all $x$ and since $F$ is nondegenerate, by \textbf{Billingsley Section 14, Lemma 5} it follows that $\beta = 1$ and $\alpha=0$.

Therefore,
\[
\lim_{n \to \infty}
-\frac{a_n' - a_n}{b_n}
=
\lim_{n \to \infty} \frac{a_n - a_n'}{b_n} 
=
\lim_{n \to \infty} \alpha_n 
= 
\alpha 
= 
0
\]
and
\[
\lim_{n \to \infty} \frac{b_n'}{b_n}
=
\lim_{n \to \infty} \beta_n 
= 
\beta
=
1,
\]
whence
\[
\frac{b_n}{b_n'} \to 1
\]
as desired.
\end{proof}
\end{document}