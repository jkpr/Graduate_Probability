% /**
%  * A template for homework files in math classes. The 
%  * packages and newcommands are a good starting point.
%  *
%  * Author: James K. Pringle
%  * E-mail: jameskpringle@gmail.com
%  * Last Changed: 26 February 2013
%  *
%  * "LaTeX countains the increasing union of MS Word"
%  */
%~~~~~~~~~~~~~~~~~~~~~~~~~~~~~~~~~~~~~~~~~~~~~~~~~~~~~~~~~%
%%%%%%%%%%%%%%%%%%%%%%%%%%%%%%%%%%%%%%%%%%%%%%%%%%%%%%%%%%%
%                                                         %
%                        PAGE SETUP                       %
%                                                         %
%%%%%%%%%%%%%%%%%%%%%%%%%%%%%%%%%%%%%%%%%%%%%%%%%%%%%%%%%%%
\documentclass[letterpaper, 12pt]{article}

% 1in margins all the way around
\usepackage[margin=1in]{geometry}

% Sets \parindent to 0 and \parskip to stretchable.
\usepackage{parskip}
% Use for bigger spaces between paragraphs.
%\parskip=1.5\baselineskip

% Set headers and footers
\usepackage{fancyhdr}
\pagestyle{fancy}
% Header
\renewcommand{\headrulewidth}{0.4pt}
\lhead{\textsc{\mathclass}}
\chead{\textsc{\today}}
\rhead{\textsc{\mynamehdr}}
% Footer
\renewcommand{\footrulewidth}{0.4pt}
\lfoot{}
\cfoot{\thepage}
\rfoot{}

% Make the space between lines slightly more generous 
% than normal single spacing, but compensate so that the 
% spacing between rows of matrices still looks normal.  
% Note that 1.1=1/.9090909...
\renewcommand{\baselinestretch}{1.1}
\renewcommand{\arraystretch}{.91}

%%%%%%%%%%%%%%%%%%%%%%%%%%%%%%%%%%%%%%%%%%%%%%%%%%%%%%%%%%%
%                                                         %
%                      USEFUL PACKAGES                    %
%                                                         %
%%%%%%%%%%%%%%%%%%%%%%%%%%%%%%%%%%%%%%%%%%%%%%%%%%%%%%%%%%%

% The classic three
\usepackage{amsmath,amsthm,amssymb}

% Define \newtheorem for use
% No numbers, labeled 'Theorem'
\newtheorem*{nthm}{Theorem}

% Not sure what this is for
\usepackage{amsfonts}

% Fancy script font
\usepackage{mathrsfs}

% Makes enumerate environment much easier to customize
% by specifying the counter
\usepackage{enumerate}

% Color
\usepackage{color}
\usepackage[usenames,dvipsnames,svgnames,table]{xcolor}

% URL links
\usepackage{hyperref}

% For inserting graphics and images
\usepackage{graphicx}
\usepackage{float}
\usepackage[footnotesize]{caption}



%%%%%%%%%%%%%%%%%%%%%%%%%%%%%%%%%%%%%%%%%%%%%%%%%%%%%%%%%%%
%                                                         %
%                   USER-DEFINED COMMANDS                 %
%                                                         %
%%%%%%%%%%%%%%%%%%%%%%%%%%%%%%%%%%%%%%%%%%%%%%%%%%%%%%%%%%%

% Make a hyperlink with colored text
\newcommand{\hrefcolor}[3]{\href{#1}{\textcolor{#3}{#2}}}

% Make a hyperlink with gray text
\newcommand{\hrefgray}[2]{\hrefcolor{#1}{#2}{Gray}}

% Make the header for the first page
\newcommand{\firstpageinfo}{
\textsf{
\begin{flushleft}
\sc \myname \\
\normalfont \mathclass \\
\professorname \\
\assignmentnumber \\
\thedate
\end{flushleft}
} \bigskip
}

% Make problem list for "title" of page
\newcommand{\problemlist}{ 
\begin{center}
\textbf{\Large \textsf{\assignmentnumber}}\\
\textit{\textsf{\problemset}}
\end{center}
\bigskip
}

%~~~~~~~~~~~~~~~~~~~~~~~~~~~~~~~~~~~~~~~~~~~~~~~~~~~~~~~~~%
%                                                         %
%               LETTERS, FUNCTIONS, AND TEXT              %
%                                                         %
%~~~~~~~~~~~~~~~~~~~~~~~~~~~~~~~~~~~~~~~~~~~~~~~~~~~~~~~~~%

% A
\newcommand{\cA}{\mathcal{A}}
\newcommand{\sA}{\mathscr{A}}
\renewcommand{\aa}{\;\text{a.a.}}
\renewcommand{\ae}{\;\text{a.e.}}
% B
\newcommand{\sB}{\mathscr{B}}
\newcommand{\cB}{\mathcal{B}}
% C
\newcommand{\cC}{\mathcal{C}}
% E
\newcommand{\E}{\mathbb{E}}
% F
\newcommand{\sF}{\mathscr{F}}
\newcommand{\cF}{\mathcal{F}}
\newcommand{\Ft}{F^\sim}
% G
\newcommand{\cG}{\mathcal{G}}
\newcommand{\sG}{\mathscr{G}}
% I
\newcommand{\io}{\;\text{i.o.}}
% N
\newcommand{\N}{\mathbb{N}}
% P
\newcommand{\cP}{\mathcal{P}}
\newcommand{\sP}{\mathscr{P}}
% Q
\newcommand{\Q}{\mathbb{Q}}
% R
\newcommand{\R}{\mathbf{R}}
\newcommand{\cR}{\mathcal{R}}
\newcommand{\sR}{\mathscr{R}}
% S
\newcommand{\cS}{\mathcal{S}}
% U
\newcommand{\cU}{\mathcal{U}}
% V
\newcommand{\var}{\text{var}}
% Z
\newcommand{\Z}{\mathbb{Z}}
% Punctuation
\newcommand{\sbs}{\;|\;} % space bar space
% Math
\newcommand{\sion}{\sum_{i=1}^n}
\newcommand{\sioi}{\sum_{i=1}^\infty}

%~~~~~~~~~~~~~~~~~~~~~~~~~~~~~~~~~~~~~~~~~~~~~~~~~~~~~~~~~%
%                                                         %
%            CHANGE THESE BASED ON THE PAPER              %
%                                                         %
%~~~~~~~~~~~~~~~~~~~~~~~~~~~~~~~~~~~~~~~~~~~~~~~~~~~~~~~~~%

% Constants for fancy header and first page info
\newcommand{\mynamehdr}{\hrefgray{http://biostat.jhsph.edu/~jpringle/}{\myname}}
\newcommand{\mathclass}{550.621 Probability}
\newcommand{\myname}{James K. Pringle}
\newcommand{\professorname}{Dr. Jim Fill}
\newcommand{\assignmentnumber}{Assignment 5}
\newcommand{\thedate}{\today}
\newcommand{\problemset}{Radon-Nikodym Theorem and fields}

%%%%%%%%%%%%%%%%%%%%%%%%%%%%%%%%%%%%%%%%%%%%%%%%%%%%%%%%%%%
%                                                         %
%                      BEGIN DOCUMENT                     %
%                                                         %
%%%%%%%%%%%%%%%%%%%%%%%%%%%%%%%%%%%%%%%%%%%%%%%%%%%%%%%%%%%
\begin{document}

% Take header off of first page
\thispagestyle{empty}

% Put in first page info (top of page)
\firstpageinfo

% Put in title for the paper
\problemlist

\section*{Part 1}
Let $P$ and $Q$ be probabilities on a measurable space $(\Omega, \cA)$.
Show that condition $2$,
\[
\lim_{\eta \to 0} \sup\{ P(A): A  \in \cA \text{ and } Q(A) < \eta \} = 0
\]
is equivalent to $2'$,
\[
\lim_{\eta \to 0} \sup\{ P(A): A  \in \cA_0 \text{ and } Q(A) < \eta \} = 0
\]
where $\cA_0$ is a field generating $\cA$.

\begin{proof}
For fixed $\eta$, since $\cA_0 \subset \cA$,
\[
\{A\in \cA_0 : Q(A) < \eta \} \subset \{A \in \cA : Q(A) < \eta\}
\]
Hence 
\[
\{ P(A): A  \in \cA_0 \text{ and } Q(A) < \eta \}
\subset
\{ P(A): A  \in \cA \text{ and } Q(A) < \eta \}
\]
Therefore
\[
0
\leq
\sup
\{ P(A): A  \in \cA_0 \text{ and } Q(A) < \eta \}
\leq
\sup
\{ P(A): A  \in \cA \text{ and } Q(A) < \eta \}
\]
Since this is true for all $\eta$,
taking the limit as $\eta \to 0$,
\[
0
\leq
\lim_{\eta \to 0}
\sup
\{ P(A): A  \in \cA_0 \text{ and } Q(A) < \eta \}
\leq
\lim_{\eta \to 0}
\sup
\{ P(A): A  \in \cA \text{ and } Q(A) < \eta \}
\]
By the assumption of condition 2, it follows by squeezing that
\[
\lim_{\eta \to 0}
\sup
\{ P(A): A  \in \cA_0 \text{ and } Q(A) < \eta \}
= 0
\]
and condition $2'$ holds. Thus condition 2 implies condition $2'$.

Next suppose that condition $2'$ holds. Note $\emptyset \in \cA_0$ since $\Omega \in \cA_0$ and $\cA_0$ is closed under complementation. Also, $\cA_0$ is closed under binary intersection. Finally since $\cA_0$ is closed under complementation, $A \subset B$ with $A, B \in \cA_0$ implies $B \setminus A = B \cap A^c \in \cA_0$. Therefore, $\cA_0$ is a semiring.
See Billingsley Section 11 for a definition of ``semiring.''

Fix $\eta > 0$. Let $A \in \cA$ such that $Q(A) < \eta$. Thus $0 < \eta - Q(A)$. Note $\sigma(\cA_0) = \cA$ and $Q$ is a p.m. (hence $\sigma$-finite) on $\cA$. By Billingsley Theorem 11.4(i), there exists an infinite\footnote{
The original theorem says ``there exists a finite or infinite disjoint sequence...'' However, since the empty set is an element of any semiring, then in the finite case with $m$ sets, define $A_n = \emptyset$ for $n > m$.
} disjoint sequence $A_1, A_2, \cdots$ of $\cA_0$ sets such that 
\[
A \subset \cup_k A_k
\] 
and 
\[
Q((\cup_k A_k) \setminus A) < \eta - Q(A).
\]
Thus
\begin{align*}
Q((\cup_k A_k) \setminus A) &< \eta - Q(A) 
\\
Q(\cup_k A_k) - Q( A) &< \eta - Q(A)  
\text{ by additivity}
\\
Q(\cup_k A_k)  &< \eta 
\end{align*}
Let $E_n = \cup_{k=1}^n A_k$, and $E = \cup_k A_k$. From the above, $Q(E) < \eta$. Since $E_n \uparrow E$, for all $n$ 
\[
Q(E_n) \leq Q(E) < \eta
\]
By the closure properties of a field, for all $n$, 
\[
E_n \in \{A \in \cA_0 : Q(A) < \eta\}.
\]
Thus for all $n$
\[
P(E_n) \leq \sup\{P(A) : A \in \cA_0 \text{ and } Q(A) < \eta\}
\]
The right-hand side does not depend on $n$, so passing to the limit
\[
\lim_{n \to \infty} P(E_n) \leq \sup\{P(A) : A \in \cA_0 \text{ and } Q(A) < \eta\}
\]
By Monotone Sequential Continuity from Below,
\[
\lim_{n \to \infty} P(E_n) = P(E) = P(\cup_k A_k)
\]
and since $A \subset \cup_k A_k$
\[
P(A) \leq P(\cup_k A_k) \leq \sup\{P(A) : A \in \cA_0 \text{ and } Q(A) < \eta\}
\]
Since $A$ is arbitrary in $\cA$ as long as $Q(A) < \eta$, then for all $A \in \cA$ with $Q(A) < \eta$
\[
0 \leq
P(A) \leq \sup\{P(A) : A \in \cA_0 \text{ and } Q(A) < \eta\}
\]
Thus
\[
0 \leq 
\sup \{P(A) : A \in \cA \text{ and } Q(A) < \eta\}
\leq
\sup \{P(A) : A \in \cA_0 \text{ and } Q(A) < \eta\}
\]
Since $\eta$ is arbitrary, take the limit as $\eta \to 0$,
\[
0 \leq 
\lim_{\eta \to 0} \sup \{P(A) : A \in \cA \text{ and } Q(A) < \eta\}
\leq 
\lim_{\eta \to 0}
\sup \{P(A) : A \in \cA_0 \text{ and } Q(A) < \eta\}
\]
By the assumption of condition $2'$, the right-hand limit is 0. Thus by squeezing,
\[
\lim_{\eta \to 0} \sup \{P(A) : A \in \cA \text{ and } Q(A) < \eta\}
=
0
\]
and condition 2 holds. Hence condition $2'$ implies condition $2$. Since implications in both directions have been proved, conditions $2$ and $2'$ are equivalent.
\end{proof}

\section*{Part 2}
Specialize to $(\Omega, \cA, Q) = ([0, 1], \text{Borels}, \text{Lebesgue})$ to deduce that a nondecreasing, right-continuous function $F$ on $[0,1]$ with $F(0) = 0$ which is \textit{absolutely continuous} in the classical sense that 
\begin{quote}
for each $\epsilon > 0$, there exists an $\eta_\epsilon > 0$ such that for every choice of $n \geq 1$ and $a_i < b_i$, $1 \leq i \leq n$, for which $\sion (b_i - a_i) < \eta_\epsilon$, one has $\sion (F(b_i) - F(a_i)) < \epsilon$
\end{quote}
necessarily is of the form $F(t) = \int_0^t X(s) ds$, $0 \leq t \leq 1$, for some $X \geq 0$.

\begin{proof}
First a few definitions. Define $\sB_0$ to be the Borel sets on $(0,1]$. Define $\sB$ to be the Borel sets on $[0,1]$.  Define $\sB^1$ to be the Borel sets on $\sR^1$, the real line.

The proof is broken into two cases.

\textit{Case 1:} $F(1) = 0$. Then since $F$ is nondecreasing, $F(t) = 0$ for $0 \leq t \leq 1$. Hence a suitable function $X(s)$ is the constant function $0$. For $0 \leq t \leq 1$,
\[
0 = F(t) = \int_0^t X(s) ds = \int_0^t 0 ds = 0
\]

\textit{Case 2:} $F(1) \neq 0$. Since $F$ is nondecreasing, $F(1) > 0$. Note by the absolute continuity of $F$, it follows that $F$ is uniformly continuous (take $n = 1$ in $\sion (b_i - a_i) < \eta_\epsilon$ and $\sion (F(b_i) - F(a_i)) < \epsilon$). Since $[0, 1]$ is an interval, and $F$ is uniformly continuous on that interval, by a result from real analysis, $F$ is bounded on that interval. Hence $F(1)$ is finite. Define
\[
G(t) =
\begin{cases}
0
&
\text{for $t < 0$}
\\
F(t)/F(1)
&
\text{for $t \in [0,1]$}
\\
1
&
\text{for $t > 1$}
\end{cases}
\]
Shortly, it is shown that $G$ is a d.f.
Note $\lim_{n \to -\infty} G(t) = 0$ and $\lim_{n \to \infty} G(t) = 1$. 
Also, $G$ is right-continuous on $[0,1)$ because it is a constant ($1/F(1)$) times $F$. 
Clearly $G$ is right-continuous on $(-\infty, 0)$. 
And since $G(1) = F(1)/F(1) = 1$, it follows that $G$ is right-continuous on $[1, \infty)$. 
Finally, since $F(1)$ is a positive constant, $F(t)/F(1)$ is nondecreasing because $F$ is nondecreasing. Therefore $G$ is nondecreasing. Thus, $G$ is a d.f.

By \textbf{Chung Theorem 2.2.4} let $\mu$ be the p.m. induced by $G$ on $(\sR^1,\sB^1)$.
Note $0 \leq G(0 -) \leq G(0) = 0$. Thus $ 0 = G(0-) = \mu((-\infty, 0))$. 
Moreover, $\mu((1, \infty)) = 1-\mu((-\infty, 1]) = 1 - G(1) = 0$. Hence $\mu([0,1]) = 1$.

Define a set function $P$ to be the restriction of $\mu$ on $\sB$.
This is valid since $\sB \subset \sB^1$.
Shortly, it is shown that $P$ is a p.m.
As shown above $P([0,1]) = 1$. 
For any $A \in \sB$, it is the case that $A \subset [0,1]$. Thus $0 \leq P(A) \leq 1$. Finally, let $A_1, A_2, \cdots$ be disjoint sets in $\sB$. Then since $\cup_{i=1}^\infty A_i \in \sB \subset \sB^1$,
\[
P(\cup_{i=1}^\infty A_i) =\mu(\cup_{i=1}^\infty A_i) = \sioi \mu(A_i) = \sioi P(A_i)
\]
Therefore, $P$ is a p.m. on $([0,1], \sB)$.

Let $\cA_0'$ be the class of finite unions of disjoint intervals $(a,b]$ contained in $[0,1]$. By \textbf{Chung Example 2.2.2}, $\cA_0'$ is a field on $(0,1]$ and generates $\sB_0$. However, $\cA_0'$ is not a field on $[0,1]$ (because $[0,1]$ is not an element of $\cA_0'$). The next task is to describe a field on $[0,1]$ that generates the Borel sets on $[0,1]$. 

Let $\cA_0$ be the class of sets of $\cA_0'$ possibly unioned with $\{0\}$. Hence $A \in \cA_0$ can be written as
\[
A = A'
\quad
\text{or}
\quad
A = \{0\} \cup A'
\] 
where $A' \in \cA'_0$ and
\[
A' = \cup_{i=1}^n (a_i, b_i]
\]
with
$0 \leq a_1 < b_1 < a_2 < b_2 < \cdots < a_n < b_n \leq 1$ and $n \geq 0$.

Notice that (using the two possible $A$ definitions above)
\[
A^c = \{0\} \cup (0, a_1] \cup (\cup_{i=1}^{n-1} (b_i, a_{i+1}]) \cup (b_n, 1]
\quad
\text{or}
\quad
A^c = (0, a_1] \cup (\cup_{i=1}^{n-1} (b_i, a_{i+1}]) \cup (b_n, 1],
\]
respectively. Hence $A^c \in \cA_0$. Next, suppose $B_1, B_2 \in \cA_0$. 
\textit{Case 1:} $\{0\}$ is not a subset of both $B_1$ and $B_2$. Then $B_1 \cap B_2 \in \cA_0'$ and hence $B_1 \cap B_2 \in \cA_0$. 
\textit{Case 2:} $\{0\}$ is a subset of both $B_1$ and $B_2$.
Then $B_1 \cap (B_2 \setminus \{0\}) \in \cA_0'$ by case 1, and 
\[
(B_1 \cap (B_2 \setminus \{0\})) \cup \{0\} =
B_1 \cap B_2 
\in \cA_0
\]
by definition of $\cA_0$. Hence $\cA_0$ is a field on $[0,1]$. 

From \textbf{Chung Example 2.2.2}, the Borel sets on $[0,1]$ are generated by the Borel sets on $(0,1]$ and the singleton $\{0\}$. In math, 
\[
\sB = \sigma(\sB_0, \{0\})
\]
Let $A \in \cA_0$. Then $A = A'$ or $A = A' \cup \{0\}$ where $A' \in \cA_0'$. 
Note $A' \in \sigma(\cA_0') = \sB_0$. Therefore, $A' \in \sB$. Moreover, $\{0\} \in \sB$.
Hence by closure of $\sigma$-fields, $A$ is an element of $\sB$. Since $A$ was arbitrary, $\cA_0 \subset \sB$. Therefore $\sigma(\cA_0) \subset \sB$.

Next, let $\sB^*$ be the trace of $\sigma(\cA_0)$ onto $(0,1]$. In math
\[
\sB^* = \{ A \cap (0,1] : A \in \sigma(\cA_0)   \}
\]
From class notes, 550.620, slide 38, $\sB^*$ is a $\sigma$-field on $(0,1]$. Since $(0,1] \in \sigma(\cA_0)$, it follows that $\sB^* \subset \sigma(\cA_0)$ by closure of $\sigma$-fields. For $A' \in \cA_0'$, it is the case that $A' \subset (0,1]$ and $A' \in \cA_0 \subset \sigma(\cA_0)$. Thus $A' \in \sB^*$. 
Since $A'$ is arbitrary, $\cA_0' \subset \sB^*$. Hence $\sB_0 = \sigma(\cA_0') \subset \sB^* \subset \sigma(\cA_0)$. Finally, since $\{0\} \in \cA_0$, it follows that  $\{0\} \in \sigma(\cA_0)$. The previous two facts show that $\sB = \sigma(\sB_0, \{0\}) \subset \sigma(\cA_0)$. Since containment has been shown in both directions, $\sB = \sigma(\cA_0)$. In words, the field $\cA_0$ on $[0,1]$ generates the $\sigma$-field of the Borel sets on $[0,1]$.

Note $Q(\{0\}) = 0$. 
Hence for $A \in \cA_0$, where $A = A'$ or $A=A' \cup \{0\}$ for some $A' \in \cA_0'$, it follows that $Q(A) = Q(A')$ or $Q(A') + Q(\{0\}) = Q(A')$.
Then by definition of $\cA_0$,
\[
\{A \in \cA_0 : Q(A) < \eta\} = \{A \in \cA_0' : Q(A) < \eta\}
\]
Note $P(\{0\}) = \mu(\{0\}) = \mu((-\infty, 0]) - \mu((-\infty, 0)) = G(0) - G(0-)  = 0$. Hence for $A \in \cA_0$, where $A = A'$ or $A=A' \cup \{0\}$ for some $A' \in \cA_0'$, it follows that $P(A) = P(A')$ or $P(A') + P(\{0\}) = P(A')$. Therefore, by the two facts above about $P$ and $Q$, it is the case that
\begin{align*}
&\quad\;\lim_{\eta \to 0} \sup\{ P(A): A  \in \cA_0 \text{ and } Q(A) < \eta \} 
\\
&=
\lim_{\eta \to 0} \sup\{ P(A): A  \in \cA_0' \text{ and } Q(A) < \eta \}
\\
&=
\lim_{\eta \to 0} \sup\{ P(\cup_{i=1}^n (a_i, b_i]): 0 \leq a_1 < b_1 < \cdots < a_n < b_n \leq 1, n \geq 0, Q(\cup_{i=1}^n (a_i, b_i]) < \eta \}
\\
&=
\lim_{\eta \to 0} \sup\{ \sion \mu((a_i, b_i]): 0 \leq a_1 < b_1 < \cdots < a_n < b_n \leq 1, n \geq 0, \sion Q((a_i, b_i]) < \eta \}
\\
&=
\lim_{\eta \to 0} \sup\{ \sion G(b_i) - G(a_i): 0 \leq a_1 < b_1 < \cdots < a_n < b_n \leq 1, n \geq 0, \sion b_i - a_i < \eta \}
\\
&=
\lim_{\eta \to 0} \sup\{ \frac{1}{F(1)}\sion F(b_i) - F(a_i): 0 \leq a_1 < b_1 < \cdots < a_n < b_n \leq 1, n \geq 0, \sion b_i - a_i < \eta \}
\end{align*}
Let $\epsilon > 0$. Then by the absolute continuity of $F$, since $\epsilon F(1) > 0$, there exists a $\delta > 0$, such that if $\sion b_i - a_i < \delta$, then $\sion F(b_i) - F(a_i) < \epsilon F(1)$. Or in other words, 
\[
\frac{1}{F(1)}\sion F(b_i) - F(a_i) < \epsilon.
\]
Notice $\epsilon$ is a bound that does not depend on the choice of the $a_i$'s and the $b_i$'s.
Therefore, as $\delta \to 0$, i.e. as the bound on $\sion b_i - a_i$ tends to $0$, it is the case that 
\[
\frac{1}{F(1)}\sion F(b_i) - F(a_i) \to 0
\] 
uniformly in the choice of 
$0 \leq a_1 < b_1 < \cdots < a_n < b_n \leq 1$ and $n \geq 0$. Hence
\begin{align*}
&\quad\;\lim_{\eta \to 0} \sup\{ P(A): A  \in \cA_0 \text{ and } Q(A) < \eta \} 
\\
&=
\lim_{\eta \to 0} \sup\{ \frac{1}{F(1)}\sion F(b_i) - F(a_i): 0 \leq a_1 < b_1 < \cdots < a_n < b_n \leq 1 \text{ and } \sion b_i - a_i < \eta \}
\\
&=
0
\end{align*}
and condition $2'$ holds. 
By the first part of this homework, condition $2$ holds. Thus by the Little Radon-Nykodym theorem, there exists $X \geq 0$ such that $P(A) = \int_A X dQ$ for all $A \in \sB$. In particular, take $A = [0, t]$, then 
\[
F(t)/F(1) = G(t) = \mu((-\infty, t]) = \mu([0,t]) = P([0,t]) = P(A) = \int_A X dQ = \int_0^t X(s) ds
\] 
Hence there exists a function, $F(1)X$, such that
\[
F(t) = \int_0^t F(1) X(s) ds
\]
for $0 \leq t \leq 1$
Furthermore, since $F(1) >0$, $F(1) X \geq 0$.
%Since $\cA_0'$ generates the Borel sets on $(0,1]$
%Since $\cA_0' \subset \cA_0$, it follows that $\sigma(\cA_0') \subset \sigma(\cA_0)$, where both $\sigma$-fields are with respect to $\Omega = [0,1]$. Since $(0,1] \in \cA_0'$, it follows that $\{0\} \in \sigma(\cA_0')$ by closure. 
\end{proof}
\end{document}